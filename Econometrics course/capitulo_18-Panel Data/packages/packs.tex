% Apartado de letras
\usepackage[utf8]{inputenc}
\usepackage[T1]{fontenc}
\usepackage[spanish]{babel}

% Aparatdo matemático
\usepackage{amsfonts}
\usepackage{amsmath}
\usepackage{amssymb}

% Apartado tipò de letra
\usepackage{libertine}
\usepackage[libertine]{newtxmath}

% Aparatdo  de lista: cambiar la enumeración
\usepackage{enumitem}

% Apartado tablas, figuras, y otros
\usepackage{multirow,booktabs,setspace,caption,multicol}
\captionsetup{skip=0pt}
	% Apartado Tablas y Figuras en APA
	\DeclareCaptionLabelSeparator*{spaced}{\\[1ex]}
	\captionsetup[table]{name = Tabla, textfont=it,format=plain,justification=justified,
		singlelinecheck=false,labelsep=spaced,skip=0pt}
	\captionsetup[figure]{name = Figura, labelsep=period,labelfont=it,justification=justified,
		singlelinecheck=false,font=doublespacing}
	% Para hacer tablas APA más fácil
	\usepackage[flushleft]{threeparttable}

% Apartado hyperres
\usepackage{hyperref}

% Apartado Tikz
\usepackage{tikz, pgfplots}
\usetikzlibrary{positioning,calc}

% Apartado de colores
\usepackage{xcolor}
\definecolor{greencool}{RGB}{30, 52, 56}

% Apartado de íconos
\usepackage{fontawesome5}
\definecolor{twit}{RGB}{29, 161, 242}

%Creación de entornos matemáticas
\newtheorem{defi}{Definición}[section]
\newtheorem{lema}{Lema}[section]
\newtheorem{teo}{Teorema}[section]
\newtheorem{coro}{Corolario}[section]

% Diversos
\usefonttheme[onlymath]{serif}
\usepackage{remreset}
\usepackage{makecell}
\usepackage{subfig}
\usepackage{float}
\usepackage{subfig}
% \graphicspath{{./fig/}} es para definir una ruta es como esl cd de stata

% Apartado justificación de textos
\usepackage{ragged2e}
\justifying
\renewcommand{\raggedright}{\leftskip=0pt \rightskip=0pt plus 0cm}

% Apartado miniframe
\makeatletter
\let\beamer@writeslidentry@miniframeson=\beamer@writeslidentry
\def\beamer@writeslidentry@miniframesoff{%
	\expandafter\beamer@ifempty\expandafter{\beamer@framestartpage}{}% does not happen normally
	{%else
		% removed \addtocontents commands
		\clearpage\beamer@notesactions%
	}
}
\newcommand*{\miniframeson}{\let\beamer@writeslidentry=\beamer@writeslidentry@miniframeson}
\newcommand*{\miniframesoff}{\let\beamer@writeslidentry=\beamer@writeslidentry@miniframesoff}
\beamer@compresstrue
\makeatother

% Apartado de Stata
\usepackage{listings}
\definecolor{codeblue}{rgb}{0.29296875, 0.51953125, 0.68359375}
\definecolor{codegreen}{rgb}{0.47265625, 0.62890625, 0.40234375}
\definecolor{codegray}{rgb}{0.95703125, 0.95703125, 0.95703125}
\definecolor{codecrimson}{rgb}{0.87109375,0.3984375,0.3984375}

\lstset{frame=tb,
	backgroundcolor=\color{codegray},
	aboveskip=3mm,
	belowskip=3mm,
	showstringspaces=false,
	columns=flexible,
	basicstyle={\ttfamily},
	numbers=left,
	numberstyle=\tiny\color{gray},
	keywordstyle=\color{codeblue},
	commentstyle=\color{codegreen},
	stringstyle=\color{codecrimson},
	breaklines=true,
	breakatwhitespace=true,
	tabsize=4,
	numbers=left,                    
	numbersep=5pt,
	frame=tlbr,framesep=4pt,framerule=0pt
}

% Apartado de block_colores
	% Some examples
	\setbeamercolor{block title}{fg=black,bg=red!20!white}
	\setbeamercolor{block body}{fg=black, bg=red!5!white}
	
	% STATA code
	\setbeamercolor{block title example}{fg=black, bg=gray!15!white}
	\setbeamercolor{block body example}{bg=codegray}