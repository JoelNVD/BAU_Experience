%====================================================================================
\section[Modelo]{Modelo de probabilidad lineal}
%====================================================================================

\subsection{Modelo de probabilidad lineal}
\begin{frame}{Modelo de probabilidad lineal}
	\begin{itemize}
		\item
		Sea $y$ una variable dicotómica. Entonces: $E(y/x)=P(y=1/x)*1+P(y=0/x)*0=P(y=1/x)*1$
		\pause
		\item
		Entonces, en este caso: $E(y/x)=\beta_{0}+\beta_{1}x_{1}+...+\beta_{k}x_{k}=P(y=1/x)$
		\pause
		\item Por lo tanto, la interpretación de $\beta_{i}$ es el cambio en la probabilidad de éxito cuando $x_{i}$ cambia
		\pause
		\item El valor que predice este modelo es la probabilidad de éxito, que tiene el inconveniente que podría estar fuera del intervalo
		esperado de [0,1].
		\pause
		\item Otro problema que se tiene es que el modelo viola el supuesto de homocedasticidad (Probar!)
	\end{itemize}
\end{frame}
