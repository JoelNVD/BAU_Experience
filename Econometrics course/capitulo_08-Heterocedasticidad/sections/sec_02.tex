%====================================================================================
\section[¿Problema?]{¿Por qué preocuparnos de la heterocedasticidad?}
%====================================================================================

\begin{frame}{¿Porqué preocuparnos de la heterocedasticidad?}
		\begin{center}
			\begin{tikzpicture}[
	declare function = {
		weibullpdf(\x,\a,\b) = (\b/\a)*(\x/\a)^(\b-1)*exp(-(\x/\a)^(\b));
	},
	declare function={
		normalpdf(\x,\mu,\sigma)=
		(2*3.1415*\sigma^2)^(-0.5)*exp(-(\x-\mu)^2/(2*\sigma^2));
	},
	hplot/.style={ycomb, mark=o, dashed}]
	\begin{axis}[scale=0.7,
		title style={at={(0.5,0)},anchor=north,yshift=3.5cm},
		title = Variance of the estimator,
		legend style={at={(0.5,-0.1)},anchor=north, yshift=-3mm},
		domain=-3:3, width=10cm, height=6cm, samples=100,
		xlabel style={at={(1,0)}, anchor=north west},
		ylabel style={rotate=-90, at={(0,1)}, anchor=south east},
		axis x line*=bottom, axis y line*=left,
		xmin = -4, xmax = 4, ymax = 0.8, ymin = -0.05,
		xtick={-4, -2, 0, 2, 4}, ytick={0, .2, .4, .6, .8},
		ymajorgrids=true]
		
		% Curvas
		\addplot [very thick, redblack] {normalpdf(x,0,1)};
		\addplot [very thick, orangeblack] {normalpdf(x,0,0.6)};
		
		% A´reas coloreadas
		\addplot[domain = -5:-2, fill = red, fill opacity = 0.5, draw = none, thick]{normalpdf(x,0,1)} \closedcycle;
		\addplot[domain = 2:5, fill = red, fill opacity = 0.5, draw = none, thick] {normalpdf(x,0,1)} \closedcycle;
		
		% Leyenda: el orden que graficas es el orden de colores
		\addlegendentry{\tiny OLS with Heteroskedasticity}
		\addlegendentry{\tiny \hspace{-1.4cm} Wrong Variance}
	\end{axis}
\end{tikzpicture}
		\end{center}
	Si usamos la varianza de homocedasticidad del estimador para hacer inferencia, podemos rechazar la hipótesis nula cuando esa de hecho es verdad (Error Tipo 1).
\end{frame}
%---------------------------------------------------
\begin{frame}{¿Porqué preocuparnos de la heterocedasticidad?}
	\begin{itemize}
		\item Aun con heterocedasticidad los estimadores MCO son insesgados y consistentes.
		\pause
		\item Sin embargo, los errores estándar de los estimadores son sesgados
		\pause
		\item Al ser sesgados los errores estándar entonces no se pueden usar los estadísticos usuales t y F para hacer inferencia
	\end{itemize}
\end{frame}
%---------------------------------------------------
\begin{frame}{¿Porqué preocuparnos de la heterocedasticidad?}
	Entonces tenemos que encontrar un estimador MELI¡.\\
	De hecho, podemos mantener nuestras estimaciones de MCO, debes saber que:	
	\begin{enumerate}
		\item MCO es aún insesgado.
		\item El problema es que podemos hacer una inferencia incorrecta (surge error tipo 1), solo lo que tenemos que hacer es calcular la verdadera varianza (asumiendo heterocedasticidad) y simplemente hacer la inferencia.
		\item Si nos interesa la predicción o el cálculo de intervalos de confianza que necesitamos para encontrar un estimador MELI. Recuerda esto:
			$$Var(\beta^{MELI})<Var(\beta^{het})$$
	\end{enumerate}
	Un estimador MELI es el estimador de ``Mínimos Cuadrados Generalizados'' (MCG).
\end{frame}