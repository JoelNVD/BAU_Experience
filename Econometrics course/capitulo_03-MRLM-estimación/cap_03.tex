%====================================================================================
% Preamble
%------------------------------------------------------------------------------------
\documentclass[10pt]{beamer}

% Essential Packages
\usepackage{lmodern}
\usepackage{booktabs}
\usepackage{tikz}
\usepackage{pgfplots}
\usepackage[accumulated]{beamerseminar}
\usepackage{graphicx}

% Apartadode Texto
\usepackage[utf8]{inputenc}
\usepackage[spanish]{babel}
\usepackage[T1]{fontenc}

% Apartado Matemático
\usepackage{amsmath}
\usepackage{amsfonts}
\usepackage{amssymb}
\usepackage{mathtools}

% Aparatado Tikz y pgfplots
\usepackage{tikz}
\usepackage{pgfplots}

\definecolor{orangeblack}{RGB}{228, 131, 10}
\definecolor{redblack}{RGB}{156, 75, 79}
\definecolor{redred}{RGB}{255, 51, 51}

% Apartado de íconos
\usepackage{fontawesome5}

% Apartado de Justificación del texto
\usepackage{ragged2e}
\justifying
\renewcommand{\raggedright}{\leftskip=0pt \rightskip=0pt plus 0cm}

% Apartado de colores
\usepackage{xcolor}

% Apartado Stata
\usepackage{listings}

\definecolor{codeblue}{rgb}{0.29296875, 0.51953125, 0.68359375}
\definecolor{codegreen}{rgb}{0.47265625, 0.62890625, 0.40234375}
\definecolor{codegray}{rgb}{0.95703125, 0.95703125, 0.95703125}
\definecolor{codecrimson}{rgb}{0.87109375,0.3984375,0.3984375}

\lstset{frame=tb,
	backgroundcolor=\color{codegray},
	aboveskip=3mm,
	belowskip=3mm,
	showstringspaces=false,
	columns=flexible,
	basicstyle={\ttfamily},
	numbers=left,
	numberstyle=\tiny\color{gray},
	keywordstyle=\color{codeblue},
	commentstyle=\color{codegreen},
	stringstyle=\color{codecrimson},
	breaklines=true,
	breakatwhitespace=true,
	tabsize=4,
	numbers=left,                    
	numbersep=5pt,
	frame=tlbr,framesep=2pt,framerule=0pt
}

% Apartado de block_colores
	% STATA code
	\newenvironment<>{Stata code}[1]{%
		\begin{actionenv}#2%
			\def\insertblocktitle{#1}%
			\par%
			\mode<presentation>{%
				\setbeamercolor{block title}{fg=black,bg=gray!15!white}
				\setbeamercolor{block body}{bg=codegray}
			}%
			\usebeamertemplate{block begin}}
		{\par\usebeamertemplate{block end}\end{actionenv}}

% Stata PACK
\input{packages/stata}

% Apartado de configuración del Beamer
\mode<presentation> {
	\usetheme{Frankfurt}
	\setbeameroption{show notes}
	\setbeamercolor{item projected}{fg=white,bg=red}
	\setbeamertemplate{footline}[frame number]
	\usefonttheme[onlylarge]{structuresmallcapsserif}
	\usefonttheme[onlysmall]{structurebold}
	\usecolortheme{beaver}
	\setbeamercovered{transparent}
	\setbeamertemplate{navigation symbols}{}
	\setbeamertemplate{blocks}[rounded][shadow=true]
}
	
	% Apartado transparencia de contenido
	\AtBeginSection[]
	{
		\begin{frame}<beamer>{Contenido}
			\tableofcontents[currentsection,currentsubsection]
		\end{frame}
	}
	\AtBeginSubsection[]
	{
		\begin{frame}<beamer>{Contenido}
			\tableofcontents[currentsection,currentsubsection]
		\end{frame}
	}
	
	% Apartado de logo
	\logo{\includegraphics[scale=.06]{fig/logo-USAT.png}}

%====================================================================================
% Body
%====================================================================================
% Title Page
%-----------
\title[Capítulo 03]{Econometría Básica}

\subtitle{Capítulo 03: Modelo de Regresión Lineal Múltiple-Estimación}

\author[José Valderrama \& Freddy Rojas]{José Valderrama \& Freddy Rojas \\
	\texttt{jtvalderrama@gmail.com \& frojasca@gmail.com} \faIcon{envelope}\\
	Universidad Católica Santo Toribio de Mogrovejo}

\date[Septiembre de 2021]{Septiembre de 2021}

%------------------------------------------------------------------------------------
% Open
%---------
\begin{document}
	\rmfamily
		\begin{frame}
			\maketitle
		\end{frame}
%------------------------------------------------------------------------------------
% Sections
%---------
	\begin{frame}{Contenido}
		\tableofcontents
	\end{frame}

%%1) Modelo de regresión lineal simple --------------
%	%====================================================================================
\section{Introducción}
%====================================================================================
\begin{frame}[fragile]
	\frametitle{Lógica}
	\begin{itemize}
		\item El método de MV estima $\mu$ bajo la siguiente lógica:
		
		\begin{enumerate}
			\item Los datos fueron generados con $N(\mu,\sigma^2=1)$
			\item Con los datos disponibles ¿cuál es el valor de $\mu$ que hace más probable que $(1)$ sea cierto? 
		\end{enumerate}
		
		\item Notar que típicamente se conoce $\mu$ y la distribución, y con esos datos se generan los número seudoaleatorios.
		\item En este caso es al revés, primero conocemos los datos, y con estos buscamos cuál fue el $\mu$ que los pudo haber generado.
		\item Máxima verosimilitud $=$ Máxima compatibilidad entre el modelo y los datos.
	\end{itemize}
\end{frame}

\subsection{Principio de máxima verosimilitud}

\subsubsection{Definición}

\begin{frame}[fragile]
	\frametitle{Máxima verosimilitud}
	\begin{itemize}
		\item Dado un conjunto de datos, el objetivo es estimar los
		parámetros de tal manera que la muestra se parezca lo más
		posible al universo.
		\item El universo queda definido por la función de distribución que se asume tienen los datos (normal,
		exponencial, lognormal, etc)
		\item El logarítmo de la verosimilitud (log likelihood) es una transformación
		monotónica, por lo tanto...
		\item Mientras la verosimilitud (`l') $\epsilon$ $[0,1]$ el log likelihood (`ll') $ \epsilon ...$
	\end{itemize}
\end{frame}

\begin{frame}
	\frametitle{Likelihood Vs Log-Likelihood}
	\begin{figure}[H]
		\begin{centering}
			\includegraphics[scale=.7]{fig/mv.eps}
		\end{centering}
	\end{figure}
\end{frame}

\subsubsection{Formalidad}

\begin{frame}
	\frametitle{Estimación bajo MV}
	\begin{itemize}
		\item Se trata de construir la función de probabilidad
		conjunta (o función de verosimilitud) de $y_1$, $y_2$,...,
		$y_n$ suponiendo que las observaciones son independientes y
		están idénticamente distribuidas (iid)
		
		$$L(\theta)=f(y_1... y_n; \theta)=\Pi_{i=1}^n f(y_i;\theta)$$
		
		$$LL(\theta)=ln(f(y_1... y_n; \theta))=\sum_{i=1}^n ln f(y_i;\theta)$$
		
		\item `Si para un determinado valor de $\theta$, la
		verosimilitud es \emph{pequeña}, es poco probable que $\theta$ sea el valor correcto que ha generado los
		datos que observamos'
	\end{itemize}
\end{frame}

\begin{frame}
	\frametitle{Estimación bajo MV}
	\begin{itemize}
		\item Por tanto tenemos que elegir $\theta$ que maximice
		$L(\theta)$. Es decir, el estimador MV satisface la CPO:
		
		$$\frac{\partial L(\theta)}{\partial \theta}|_{\theta=\hat\theta}=0$$
		
		\item o lo que es lo mismo
		
		$$\frac{\partial log (L(\theta))}{\partial
			\theta}|_{\theta=\hat\theta}=0$$
		
		\item Y la condición de segundo orden...
	\end{itemize}
\end{frame}

\begin{frame}
	\frametitle{Matrices relacionadas a la función Log( L)}
	\begin{itemize}
		\item \textbf{Hessiana (H):} Es una matriz cuadrada (k x k) de
		las segundas derivadas de $log(L(\theta;y))$ con respecto a
		$\theta$: 
		
		$$H(\theta)=\frac{\partial ^2 Log(L(\theta;y))}{\partial \theta \partial \theta'}$$
		
		\item \textbf{Score (S):} Es una gradiente de (k x 1) de $log(L(\theta;y))$ con respecto a $\theta$: 
		$$S(\theta)=\frac{\partial Log(L(\theta;y))}{\partial \theta}$$
		
		Notar que cuando $\theta=\theta_{MV}$ $\Longrightarrow$ $S(\theta_{MV})=0$
	\end{itemize}
\end{frame}

\begin{frame}
	\begin{itemize}
		\item \textbf{Matriz de información $I(\theta)$:} Indica el
		grado de curvatura
		\begin{eqnarray*}
			I(\theta) &=& E[-\frac{\partial^2 Log(L)}{\partial \theta \partial
				\theta'}]
		\end{eqnarray*}
		
		Notar que mientras menos curvatura tenga la función de
		verosimilitud (el caso extremo es una línea recta) existirá
		mayor varianza en el estimador analizado pues:
		
		\begin{eqnarray*}
			Var(\theta)&=&[I(\theta)]^{-1}
		\end{eqnarray*}
		
	\end{itemize}
\end{frame}


\begin{frame}
	\frametitle{Ventajas y desventajas}
	\begin{itemize}
		\item \textbf{Ventaja:} El estimador MV (ML=maximum likelihood) tiene
		propiedades asintóticas óptimas entre todos los estimadores
		consistentes y normales asintóticamente.
		\item \textbf{Desventajas:}
		\begin{itemize}
			\item El estimador
			ML depende de forma importante de los supuestos sobre la
			distribución.
			\item El estimador MV tiene propiedades mediocres en
			muestras pequeñas.
		\end{itemize}
	\end{itemize}
\end{frame}

\subsection{Pruebas asintóticas}

\subsubsection{Test LR}

\begin{frame}
	\frametitle{Pruebas asintóticas}
	\begin{itemize}
		\item En econometría a menudo se plantean restricciones al modelo
		respecto a uno o más parámetros con el fin de indagar si el modelo
		es consistente con la restricción. 
		\item Por ejemplo $\beta_0=1$ o
		$\alpha+\beta=1$ en el contexto de una función de producción del
		tipo Cobb-Douglas (Retorno a escala constante). 
		\item El modelo que se
		estima imponiendo la restricción precisamente se conoce como
		modelo restringido. 
	\end{itemize}
	
	En lo que sigue se discuten tres pruebas
	asintóticas equivalentes que evalúan con procedimientos distintos
	la consistencia de una restricción.
\end{frame}

\begin{frame}
	\frametitle{Likelihood Ratio Test (LR)}
	Por sus siglas en inglés también se conoce como el ``test LR''. En
	este tipo de pruebas se requiere la estimación restringida y sin
	restringir:
	\begin{itemize}
		\item Modelo restringido: $y=\beta_0$
		\item Modelo no restringido: $y=\alpha_0+\alpha_1 x_1+\alpha_2 x_2$
		\item ambos se estiman por MV donde lo que se quiere analizar
		es si la hipótesis nula conjunta de que los coeficientes de
		las variables que acompañan a las variables son iguales a
		cero...
	\end{itemize}
	\begin{eqnarray*}
		LRT &=& 2[Log L(\hat\theta^nr)-Log L(\theta^{r})] \sim \chi^2_q
	\end{eqnarray*}
	
	Para su estimación requiere tanto de los estimadores restringidos
	como no restringidos.
\end{frame}


\subsubsection{Test de Wald}

\begin{frame}
	\frametitle{Test de Wald}
	
	Piense en la siguiente restricción matricial: $R\beta=r$ donde $R$
	es una matriz de m x k y $\beta$ es una matriz de k x 1. Sea
	$g(\beta)=R\beta-r$, si se sabe que $g(\hat\beta^R)=0$, lo que se
	pregunta el test de Wald es ¿$g(\hat\beta_{MV})=0$? es decir, se
	reemplazan los $\hat\beta_{MV}$ en la restricción:
	
	\begin{itemize}
		\item Si $g(\hat\beta_{MV})$ tiende a $0$ se acepta la
		restricción.
		\item Si $g(\hat\beta_{MV})$ no tiende a $0$ no se acepta la
		restricción.
	\end{itemize}
	
	El estadístico necesario para la prueba es:
	
	\begin{eqnarray*}
		W &=& (R\hat\beta_{MV}-r)'\{ Var (R\hat\beta_{MV}-r) \}^{-1} (R\hat\beta_{MV}-r) \\
		W &=& g(\hat\beta_{MV})'\{ \frac{\partial g}{\partial \beta} [I(\theta)]^{-1}  \frac{\partial g}{\partial \beta} \}^{-1} g(\hat\beta_{MV}) \sim \chi^2_q
	\end{eqnarray*}
	
	Para su estimación requiere sólo de los estimadores no
	restringidos.
	
\end{frame}

\subsubsection{Test Multiplicadores de Lagrange (LM)}

\begin{frame}
	\frametitle{Test de Multiplicadores de Lagrange}
	
	Se basa en la matriz score eficiencia $(S(\theta))$, si se sabe
	que $S(\hat\beta_{MV})=0$, lo que se pregunta el test LM es
	¿$S(\hat\beta^R)=0$? es decir, se reemplazan los $\hat\beta^R$ en
	el score:
	
	\begin{itemize}
		\item Si $S(\hat\beta^R)$ tiende a $0$ se acepta la
		restricción.
		\item Si $g(\hat\beta^R)$ no tiende a $0$ no se acepta la
		restricción.
	\end{itemize}
	
	\begin{eqnarray*}
		LM &=& S(\hat\theta^R)'[I(\theta)]^{-1} S(\hat\theta^R) \sim \chi^2_q
	\end{eqnarray*}
	
	Para su estimación requiere sólo de los estimadores restringidos.
\end{frame}

\subsection{MCO Vs MV}

\subsubsection{Modelo de regresión lineal simple}

\begin{frame}
	\frametitle{MRS usando MV}
	
	Sea el siguiente modelo de regresión lineal simple, en su versión
	poblacional:
	
	\begin{eqnarray}
		Y_i &=& \beta_0+\beta_1 X_i+\mu_i
	\end{eqnarray}
	
	Donde $\mu_i \sim N(0,\sigma^2)$, por lo tanto, la probabilidad de
	que el error $\mu_i$ provenga de la distribución normal es:
	
	\begin{eqnarray}
		f(\mu_i,\beta_0,\beta_1,\sigma) &=&
		\frac{1}{\sqrt{\sigma^2\pi}}\exp^{-\frac{1}{2}(\frac{\mu_i}{\sigma})^2}
	\end{eqnarray}
\end{frame}

\begin{frame}
	
	Probabilidad conjunta
	
	\begin{table}
		\centering
		\begin{tabular}{c|c}
			% after \\: \hline or \cline{col1-col2} \cline{col3-col4} ...
			N° de Observación & f \\
			\hline
			1 & $f(\mu_1,\beta_0,\beta_1,\sigma)$ \\
			2 & $f(\mu_2,\beta_0,\beta_1,\sigma)$ \\
			. & . \\
			. & . \\
			. & . \\
			n & $f(\mu_n,\beta_0,\beta_1,\sigma)$ \\
			\hline \\
			Prob. Conjunta & $\pi_i^n f(\mu_i,\beta_0,\beta_1,\sigma)$ \\
		\end{tabular}
	\end{table}
\end{frame}

\begin{frame}
	En una muestra de tamaño $n$ la probabilidad individual de que
	cada observación provenga de una distribución normal son mostrados
	en la tabla 1. La probabilidad conjunta, asumiendo independencia,
	es la productoria de todas las probabilidades:
	
	\begin{eqnarray}
		\pi_i^n f(\mu_i,\beta_0,\beta_1,\sigma) &=&L=
		\frac{1}{\sigma^n}\frac{1}{(\sqrt{2\pi})^n}\exp^{-\frac{1}{2\sigma^2}\sum_{i=1}^n\mu_i^2}
	\end{eqnarray}
	
	Con lo cual el logaritmo de la verosimilitud (LL) queda definido
	como:
	
	\begin{eqnarray*}
		LL &=&
		Ln(1)-Ln(\sigma)^n-Ln(\sqrt{2\pi})^n-\frac{1}{2\sigma^2}\sum\mu_i^2
	\end{eqnarray*}
	
	Maximizar la expresión anterior es lo mismo que maximizar la
	siguiente función:
	
	\begin{eqnarray}
		LL &=&-n Log(\sigma)-\frac{1}{2\sigma^2}\sum\mu_i^2
	\end{eqnarray}
\end{frame}

\begin{frame}
	\emph{La CPO} de la función es dado por el siguiente arreglo:
	
	\[ S(\theta) = \left( \begin{array}{cc}
		\frac{\partial LL}{\partial \beta_0}  \\
		\frac{\partial LL}{\partial \beta_1}  \\
		\frac{\partial LL}{\partial \sigma} \end{array} \right)=\left( \begin{array}{cc}
		\frac{1}{\sigma^2}\sum(Y_i-\beta_0-\beta_1 X_i)  \\
		\frac{1}{\sigma^2}\sum(Y_i-\beta_0-\beta_1 X_i)X_i  \\
		\frac{-n}{\sigma}+\frac{1}{\sigma^3}\sum\mu_i^2 \end{array} \right)=\left( \begin{array}{cc}
		0  \\
		0  \\
		0 \end{array} \right).\]
\end{frame}

\begin{frame}
	Nótese que los ecuaciones que se obtienen de las primeras dos
	filas de los vectores son las mismas ecuaciones normales que se
	obtienen cuando se resuelve el problema de MCO, por tanto los
	$\beta$s que resuelven el problema de MV son los mismos que los
	que se obtienen bajo MCO. La tercera fila del vector $S(\theta)$,
	permite conocer la dispersión de $\mu_i$:
	
	\begin{eqnarray*}
		\sigma^2_{MV} &=& \frac{\sum \hat{\mu_i}^2}{n}
	\end{eqnarray*}
	
	Que como se sabe es un estimador sesgado de la
	varianza\footnote{El estimador insesgado es: $\frac{\sum
			\hat{\mu_i}^2}{n-k}$, donde $k$ son todos los parámetros de la
		regresión a estimar incluyendo al intercepto}, aunque el sesgo se
	disipa cuando la muestra ($n$) es grande.
	\bigskip
	
	La solución al problema todavía esta incompleto, falta demostrar
	que la Hessiana es una matriz definida negativa.
\end{frame}

\begin{frame}
	Los resultados anteriores pueden ser generalizados para el caso de
	más de un regresor o covariado, la estimación en este caso viene
	dado por:
	
	\begin{eqnarray*}
		\hat{\beta} &=& (X'X)^{-1}(X'Y) \\
		\hat{\sigma}^2 &=& n^{-1}\hat{\mu}'\hat{\mu}
	\end{eqnarray*}
	
	que se obtienen luego de maximizar la función log-likelihood:
	
	\begin{eqnarray}
		LL(\theta; Y|X) &=&
		-\frac{n}{2}ln(2\pi)-\frac{n}{2}ln(\sigma^2)-\frac{1}{2\sigma^2}(Y-XB)'(Y-XB)
	\end{eqnarray}
\end{frame}

\subsubsection{Comparación}
\begin{frame}{Comparación}
	
	\begin{tabular}{|l|l|}
		\hline
		% after \\: \hline or \cline{col1-col2} \cline{col3-col4} ...
		\textbf{MCO} & \textbf{MV} \\
		\hline
		Función a optimizar:........... & Función a optimizar:........... \\
		Criterio de optimización:........... & Criterio de optimización:........... \\
		Rest. Vs No Rest:........... & Rest. Vs No Rest:........... \\
		Estadístico:...........& Estadístico:........... \\
		Interp. Est:........... & Interp. Est:........... \\
		P-Value:........... & P-Value:........... \\
		Bondad de ajuste:............. & Bondad de ajuste:........... \\
		\hline
	\end{tabular}
\end{frame}
%%2) Modelo de regresión lineal múltiple ------------
%	%====================================================================================
\section{¿Por qué estudiar series de tiempo?}
%====================================================================================

\begin{frame}{Porqué estudiar series de tiempo?}
	\begin{itemize}
		\item Para pronosticar
			\begin{itemize}
				\item ¿Cuánto será la inflación el próximo año?
				\item ¿Cuánto será el volumen de ventas de una empresa X?
			\end{itemize}
		\item Para estimar efectos causales dinámicos
			\begin{itemize}
				\item Si el banco central incrementa la tasa de interés, cuál será el impacto inmediato y el efecto sobre la inflación después de dos meses?
			\end{itemize}
	\end{itemize}
\end{frame}
%---------------------------------------------------
\begin{frame}{Ejemplos de modelos de series de tiempo}
	\begin{itemize}
		\item Modelo estático que relaciona variables contemporáneas:
			$$c_t = \beta_0 + \beta_1PBI_t + u_t$$
		\item Modelos que permiten que una o más variables afecten a y con rezagos:
			$$c_t = \alpha_0 + \delta_0 PBI_t + \delta_1 PBI_{t-1} + \delta_2 PBI_{t-2} + u_t$$
		\item \textcolor{red}{Modelos Univariados:}
			$$\textcolor{red}{c_t = \alpha_0 + \delta_0 c_{t-1} + \delta_2 c_{t-2} + \ldots + u_t}$$
		\item En general:
			$$c_t = \alpha_0 + \delta_0 PBI_t + \delta_1 PBI_{t-1} + \delta_2 c_{t-1} + u_t$$
	\end{itemize}
\end{frame}
%---------------------------------------------------
\begin{frame}{Ejemplos de modelos de series de tiempo}
	\begin{itemize}
		\item En lo que queda de clases nos vamos a concentrar en la predicción basado en el pasado de la variable que queremos predecir:
		\item \textcolor{red}{Modelos Univariados:}
			$$\textcolor{red}{c_t = \alpha_0 + \delta_0 c_{t-1} + \delta_2 c_{t-2} + \ldots + u_t}$$
	\end{itemize}
\end{frame}
%%3) Interpretación ---------------------------------
%	%====================================================================================
\section{Análisis de las series}
%====================================================================================

\begin{frame}{Análisis de las series}
	\begin{itemize}
		\item La mayoría de series de tiempo económicas se caracterizan por estar compuestas por una tendencia, un comportamiento estacional y un componente irregular:
			$$y = \text{tendencia} + \text{estacional} + \text{irregular}$$
		\item Luego de quitar las dos primeras, el problema se reduce a modelar el comportamiento irregular, que puede ser ``estacionario'' o contener al menos una ``raíz unitaria''.
	\end{itemize}
\end{frame}
%---------------------------------------------------
\begin{frame}{Estudiando el componente irregular. Procesos Estocásticos Estacionarios}
	\begin{itemize}
		\item  Una serie se define como estacionaria si los momentos de primer y segundo orden de dicho proceso estocástico son invariantes en el tiempo.
		\item Estos momentos incluyen la esperanza (media) y varianza de la serie, pero también las covarianzas y correlaciones entre los valores rezagados de la misma.
	\end{itemize}
\end{frame}
%---------------------------------------------------
\begin{frame}{Estudiando el componente irregular. Procesos Estocásticos No Estacionarios (Raíz unitaria)}
	\begin{itemize}
		\item En el caso de las series estacionarias estas tienen la característica de que sus valores oscilan alrededor de la media. Así, si una variable se desvía del valor de su media existen fuerzas que hacen que la serie retorne a su media. \textcolor{red}{La estacionariedad nos dice que el pasado es relevante!}
		\item El caso de la raíz unitaria es totalmente opuesto: los shocks que puedan afectar a la serie en determinado momento la desviarán por un lapso indeterminado de su valor medio. Por ello se dice que estos procesos tienen memoria larga y la serie deambula alrededor de su media. \textcolor{red}{La presencia de raíz unitaria significa que la serie no puede ser predicha usando su pasado}
	\end{itemize}
\end{frame}
%---------------------------------------------------
\begin{frame}{Serie estacionaria y no estacionaria}
	\centering
		\begin{figure}
			\includegraphics[width = 0.9\linewidth]{fig/figure3.jpg}
		\end{figure}
\end{frame}
%---------------------------------------------------
\begin{frame}{Descomposición de series}
	\begin{itemize}
		\item Sobre una serie temporal \textbf{$Y_t$} podemos identificar una serie de componentes básicos que se denominan respectivamente como:
		\item \textbf{TENDENCIA}: Movimientos de larga duración que se mantienen durante todo el periodo de observación.
		\item \textbf{ESTACIONALIDAD}: Movimiento que se produce, dentro de un periodo anual, por motivos no estrictamente económicos  como climáticos.
		\item \textbf{IRREGULARIDAD}: Movimientos erráticos que pueden o no ser predichos dependiendo de la característica de esta.
	\end{itemize}
\end{frame}
%---------------------------------------------------
\begin{frame}
	\centering
		\begin{figure}
			\includegraphics[width = 0.9\linewidth]{fig/figure4.jpg}
		\end{figure}
\end{frame}

%%4) MCO múltiple: Alternativa de estimación --------
%	%====================================================================================
\section[Poisson]{El modelo de regresión de Poisson}
%====================================================================================
\begin{frame}
	Principales caracter\'{i}sticas:
		\begin{itemize}
			\item El modelo Poisson está asociado a valores no negativos enteros (espacio $\mathbb{Z}$) de la variable dependiente, además de tener  una característica discreta y asociada a un conteo de casos.
			\item Por ejemplo, número de niños nacidos en el año; número de crímenes por semana; número de patentes por semana, número de casos Covid-19 en un día. 
		\end{itemize}
	El modelo Poisson puede expresarse en el valor esperado de la función exponencial:
		\begin{equation}\label{eqn:poiss}
			E(y|x_1,x_2, \dots, x_k)=\exp(\beta_0+\beta_1x_1+\dots+\beta_kx_k)
		\end{equation}
	Taking logs
		\begin{equation}
			\log (E(y|x_1,x_2, \dots, x_k))=\beta_0+\beta_1x_1+\dots+\beta_kx_k \equiv \boldsymbol{x\beta}
		\end{equation}
\end{frame}
%---------------------------------------------------
\begin{frame}{La función Log-likelihood}
	La función de distribución Poisson está definida por la probabilidad condicional en $\lambda$ para cada $h$:
		\begin{align*}
			P(Y=h | \lambda)=\frac{\exp(-\lambda)\lambda^h}{h!}
		\end{align*}
	Así modelamos en cambio la función de likelihood $P(y=h | \lambda)$
		\begin{align*}
			P(Y=h | \lambda)=\frac{\exp(-\exp(\boldsymbol{x\beta}))\exp(\boldsymbol{x\beta})^h}{h!}
		\end{align*}
	taking logs:
		\begin{align*}
			\log P(Y=h | \lambda)=-\exp(\boldsymbol{x\beta})+h \boldsymbol{x\beta}-\log (h!)
		\end{align*}
\end{frame}
%---------------------------------------------------
\begin{frame}{La función Log-likelihood}
	Considerando $h=y_i$ para una observación en particlar $i$.
		\begin{align*}
			\log P(Y=y_i | \lambda)=-\exp(\boldsymbol{x\beta})+y_i \boldsymbol{x\beta}-\log (y_i!)
		\end{align*}
	Sumando a través de las observaciones en la muestra:
		\begin{align*}
			L( \boldsymbol{\beta})=\sum_{i}^n\{y_i \boldsymbol{x\beta} - \exp(\boldsymbol{x\beta}) -\log (y_i!)\}
		\end{align*}
	Los parámetros contenidos en el vector $\beta$ se obtienen maximizando la función de log likelihood.
\end{frame}
%---------------------------------------------------
\begin{frame}{Partial effects}
	Calculando el porcentaje de cambio en el esperado condicional de $y$ dado un cambio en la variable $x$ (una semi-elasticidad):
		\begin{equation}
			\%\Delta (E(y|x_1,x_2, \dots, x_k))\approx100\beta \Delta x_j
		\end{equation}
	Siendo más precisos el cambio porcentual de $E(y|x)$ ante un cambio discreto en $x$
		\begin{equation}
			\Big(\frac{\exp(\beta_0+\beta_1x_1+\dots+\beta_k(c_k+1))}{\exp(\beta_0+\beta_1x_1+\dots+\beta_k(c_k))}-1\Big)\times 100 \equiv \Big(\exp(\beta_k)-1\Big) \times 100
		\end{equation}
	Donde el cambio discreto es igual a uno (1).
\end{frame}
%---------------------------------------------------
\begin{frame}{Partial effects}
	Se puede comparar los efectos parciales del modelo Poisson con el modelo logit y probit. Para esto se calcula $\frac{\partial E(y|x)}{\partial x_j} $ en la expresión (\ref{eqn:poiss}):
		\begin{align}
			\frac{\partial E(y|x)}{\partial x_j} =\exp(\beta_0+\beta_1x_1+\dots+\beta_kx_k) \beta_j
		\end{align}
	Tenemos que en el APE, el factor de escala es: $\sum_{i=1}^n \exp(\beta_0+\beta_1x_{i1}+\dots+\beta_kx_{ik}) $. Así entonces:
		\begin{align}
			\frac{\partial E(y|x)}{\partial x_j} =\overline{y}  \beta_j
		\end{align}
\end{frame}
%---------------------------------------------------
\begin{frame}{Important issue}
	Aunque el modelo Poisson es una elección natural para modelar datos de conteo. Tiene un ``important issue''.
		\begin{align}
			Var(y|x)=E(y|x)
		\end{align}
	En otras palabras es muy restrictivo. Aquí una pregunta de rigor: ?`Cuáles son las consecuencias? !`No muy serias! Si la distibución Poisson ``does o does not fit the data'', los $\beta$'s son aún consistentes y asintóticamente normal. Lo último es análogo al estimador OLS, el cual es consistente y asintóticamente normal si la suposición de normalidad se mantenga o no.
\end{frame}
%---------------------------------------------------
\begin{frame}{Important issue}
	Se puede hacer el modelo aún más flexible:
		\begin{align}
			Var(y|x)=\sigma^2 E(y|x)
		\end{align}
	El parámetro $\sigma^2$ es desconocido. Si $\sigma^2>1$ entonces existe evidencia que el proceso generador de datos tiene una varianza de ``overdispersion''; si $\sigma^2<1$ entonces la varianza tiene la característica de ``underdispersion''.
\end{frame}
%---------------------------------------------------
\begin{frame}{!Vayamos a STATA!}
	Exploremos la siguiente base de datos en la página de \href{http://fmwww.bc.edu/ec-p/data/wooldridge/datasets.list.html}{\textcolor{cyan}{Wooldridge}}. Utilizar el siguientes comando ``bcuse crime1''. Una descripción de las variables en la siguiente figura:
		\begin{figure}[htbp]
			\hspace*{+1cm} 
			\centering
				\includegraphics[width=0.63\linewidth]{fig/crime1} % requires the graphicx package
			\label{crime1}
			%  \caption{Source: own elaborati''}
		\end{figure} 
\end{frame}
%%5) Recuento de los supuestos del modelo de MCO ----
%	%====================================================================================
\section[Censurado y truncado]{Modelos de regresión censurados y truncados}
%====================================================================================


\begin{frame}{A modo de repaso}
	\begin{itemize}
		\item Usamos probit y logit para una respuesta binaria.
		\item Usamos Tobit para una solución de esquina (``corner solution outcome'')   
	\end{itemize}
	Usualmente otro rasgo de los datos cae en la misma categor\'{i}a de variables restringidas por un valor; en este caso estamos hablando de una variable censurada.
\end{frame}
%---------------------------------------------------
\begin{frame}{El modelo de datos censurados}
	¿Cómo surge una variable censurada?
		\begin{itemize}
			\item \textit{Survey design}. Es un caso de missing data (en la variable dependiente). Por ejemplo, demanda por tickets en eliminatorias para los últimos cupos al mundial.
			\item En algunos casos restricciones institucionales (Wooldridge).	   
		\end{itemize}
\end{frame}
%---------------------------------------------------
\begin{frame}{El modelo de datos censurados}
	\begin{itemize}
		\item ¿Cúal es el rasgo del modelo de variable censurada? Las unidades son observables y proveen información de las variables independientes ($X$'s); pero la información sobre la variable dependiente esta ausente (variable omitida). Es de conocimiento el valor de corte. Este valor puede ser superior (\textit{right censoring}) o inferior (\textit{left censoring}).
		\item Unidades escogen opciones como ``mas de 50 000 dólares'' ( threshold); se observan datos menos de 50 000 dólares.
		\item La misma respuesta del \textit{threshold} para muchas observaciones $i$.
	\end{itemize}
\end{frame}
%---------------------------------------------------
\begin{frame}
	\begin{figure}[htbp]
		\hspace*{+1cm} 
		\centering
			\includegraphics[width=0.65\linewidth]{fig/censored-model} % requires the graphicx package
		\label{censored}
		\caption{Source: Heij et. al 2004 ``Econometric Methods with Applications in Business and Economics''}
	\end{figure} 
\end{frame}
%---------------------------------------------------
\begin{frame}{El modelo de datos censurados}
	El modelo censurado puede expresarse de la siguiente manera:
		\begin{align}
			y_i=& \boldsymbol{x_i\beta}+u_i, \ u_i |  \ x_i, c_i \ \sim Normal(0,\sigma^2) \\
			w_i=&\min(y_i,c_i)
		\end{align}
	$u_i$ is independent of $c_i$
\end{frame}
%---------------------------------------------------
\begin{frame}
	¿Cuál es el problema de estimar este modelo por OLS? Las razones son similares como en el caso del modelo TOBIT.
		\begin{itemize}
			\item Los $\beta$'s son inconsistentes.	
		\end{itemize}
	Sin embargo existe un punto importante; en el modelo TOBIT, se esta modelando comportamiento óptimo de los individuos ($y\equiv$ consumo de alcohol), y en el modelo censurado se tiene un problema en el método de recolección porque (por alguna razón) una porción de los datos son censurados (no observados).
\end{frame}
%---------------------------------------------------
\begin{frame}{El modelo de datos censurados}
	Exploremos la siguiente base de datos en la página de \href{http://fmwww.bc.edu/ec-p/data/wooldridge/datasets.list.html}{\textcolor{cyan}{Wooldridge}}. Utilizar el siguientes comando ``bcuse recid''. Una descripción de las variables en la siguiente figura:
	\begin{figure}[htbp]
		\hspace*{+1cm} 
		\centering
			\includegraphics[width=0.53\linewidth]{fig/recid} % requires the graphicx package
		\label{recid}
		\caption{recid.des in Wooldridge's datasets}
	\end{figure} 
\end{frame}
%---------------------------------------------------
\begin{frame}{El modelo de datos truncados}
	\begin{itemize}
		\item Un modelo de datos censurados es aplicable cuando se tiene observaciones de las unidades y la información es parcial; es decir, se tiene información de las variables independientes pero no de la variable $y$.
		\item Usamos Tobit para una solución de esquina (``corner solution outcome'').
		\item Un modelo truncado tiene la característica de excluir (basado en el valor de $y$ ). No se tiene una submuestra aleatoria. Sin embargo, tenemos conocimiento de la regla de exclusión. Esta regla esta determinada si $y$ esta por encima o por debajo de cierto valor (o ``threshold'').   
	\end{itemize}
\end{frame}
%---------------------------------------------------
\begin{frame}{El modelo de datos truncados}
	?`Como surge o se identifica un modelo de datos truncados?
		\begin{itemize}
			\item El investigador presta atención a una submuestra de la población (quizas debido a costos de muestreo, ver Wooldridge). 		   
			\item Hay que enfatizar que la estimación por OLS es eficiente cuando la muestra seleccionada es aleatoria.
		\end{itemize}
	Ejemplos: Hausman y White (1977) usan data de impuestos negativos a la renta como determinante  de las ganancias individuales/familiares. El estudio solo incluía familias con renta 1.5 veces la linea de pobreza.
\end{frame}
%---------------------------------------------------
\begin{frame}{El modelo de datos truncados}
	El modelo de datos truncados puede expresarse de la siguiente manera;
		\begin{equation}
			y=\boldsymbol{x\beta}+u, u \ | \ x, \ \sim Normal(0,\sigma^2)
		\end{equation}
	y el set de datos ($y,x$) es observado solo si $y\ge c_i$ donde el ``threshold'' depende de variables x. Por ejemplo, Haussman y white (1977) definen $c_i$ como el tama\~no de la familia.  
\end{frame}
%---------------------------------------------------
\begin{frame}
	La funcion de distribución gráficamente luce de la siguiente manera;
		\begin{figure}[htbp]
			\hspace*{+1cm} 
			\centering
				\includegraphics[width=0.70\linewidth]{fig/truncated-model} % requires the graphicx package
			\label{trunc}
			\caption{Source: Chumacero, R, 2003: ``Limited dependent variable'' handout}
		\end{figure} 
\end{frame}
%---------------------------------------------------
\begin{frame}{El modelo de datos truncados}
	La funcion de distribución formalmente luce de la siguiente manera;
		\begin{align}
			g(y|x_i,c_i)=&f(y|x_i\beta,\sigma^2)/F(c_i|x_i,\beta,\sigma^2), y\le c_i
		\end{align}
	donde $f(y|x_i\beta,\sigma^2)$ denota la función de densidad normal y $F(c_i|x_i,\beta,\sigma^2)$ es la función acumulada evaluada en el threshold $c_i$.
	En pocas palabras, se re-pondera dividiendo la función de densidad normal por la acumulada para que la nueva función de densidad sume el valor de uno (1) sobre el dominio de los datos.
	Luego, se toman logs y se estima por ML (es decir, se maximiza la función ($g(y|x_i,c_i$) con los datos observados).
\end{frame}
%---------------------------------------------------
\begin{frame}{!`Vayamos a STATA!}
	Tengamos en cuenta un estudio que tiene el objetivo de modelar el desempe\`no académico como una función de las destrezas de lenguaje (``language skills'') y el tipo de programa en el que los estudiantes se han matriculado. Se requiere analizar a los estudiantes que tengan un desempe\~no mínimo de 40. Tengamos en cuenta la siguiente base de datos en formato STATA:
	\\
	use https://stats.idre.ucla.edu/stat/stata/dae/truncreg, clear
\end{frame}


%%6) Varianza de los estimadores --------------------
%	%====================================================================================
\section{Modelando el componente irregular}
%====================================================================================

%------------------------------------------------------------------------------------
\subsection{Modelos AR, MA y ARMA}
%------------------------------------------------------------------------------------

%---------------------------------------------------
\subsubsection{Proceso Autorregresivo (AR)}
%---------------------------------------------------

\begin{frame}{Proceso Autorregresivo $(AR)$}
	Definimos un \textcolor{red}{proceso autorregresivo de primer orden $AR(1)$} omo un proceso aleatorio que responde a una expresión del tipo
		$$Z_t = \rho_0 + \rho_1 Z_{t-1} + a_t \textup{ o bien } \breve{Z}_{1} = \rho_1\breve{Z}_{t-1}+a_t \textup{con} \breve{Z}_{t} = Z_t - \rho_0$$
	Para que el proceso $AR(1)$ sea estacionario se debe cumplir que $-1<\rho_1<1$ para que $\sigma_{Z}^2$ finita y no negativa.
		$$Var(\breve{Z}_{t}) = \sigma_{Z}^2 = \rho_1^2\sigma_{Z}^2 + \sigma_{a}^2 = \frac{\sigma_{a}^2}{1-\rho_1^2}$$
	Los procesos autoregresivos pueden generalizarse al orden $p$ \textcolor{red}{$AR(p)$}  sin más que añadir términos retardados en la expresión general.
		$$Z_t = \rho_0 + \rho_1 Z_{t-1} + \rho_2 Z_{t-2} + \ldots \rho_p Z_{t-p} + a_t$$
\end{frame}
%---------------------------------------------------
\begin{frame}{Simulación de dos procesos $AR(1)$}
	\centering
	\begin{figure}
		\includegraphics[width = 0.99\linewidth]{fig/figure8.jpg}
	\end{figure}
\end{frame}
%---------------------------------------------------
\begin{frame}{Función de autocorrelación (FAC) de un $AR(1)$}
	$$\widehat{\rho}_k = \frac{\gamma_k}{\gamma_0}=\phi^{k}; \textup{ donde } \phi \textup{ es el coeficiente que acompaña a } Z_{t-1}$$
	
	Por último, la función de autocorrelación parcial se corta de forma abrupta después del primer rezago.\\
	
	Es fácil ver por qué del corte. Las autocorrelaciones parciales son precisamente los últimos coeficientes en una regresión, por lo que en un proceso $AR(1)$ los coeficientes de los rezagos más largos son cero.	
\end{frame}

%---------------------------------------------------
\subsubsection{Medias móviles (MA)}
%---------------------------------------------------
\begin{frame}{Medias móviles $(MA)$}
	Definimos una \textit{media móvil} de primer orden $MA(1)$ como un proceso aleatorio que responde a una expresión del tipo
		$$Z_t = a_t + \theta_1a_{t-1}$$
	Los procesos de medias móviles son estacionarios pueden generalizarse al orden $q$ $MA(q)$ sin más que añadir términos retardados en la expresión general.
		$$Z_t = a_t + \theta_{}1a_{t-1}+\theta_{2}a_{t-2} + \ldots + \theta_{q}a_{t-q}$$
\end{frame}
%---------------------------------------------------
\begin{frame}{Simulación de dos procesos $AR(1)$}
	\centering
	\begin{figure}
		\includegraphics[width = 0.99\linewidth]{fig/figure9.jpg}
	\end{figure}
\end{frame}
%---------------------------------------------------
\begin{frame}{Función de autocorrelación (FAC) de un $MA(1)$}
	\begin{equation*}
		\widehat{\rho}_k = \frac{\gamma_k}{\gamma_0} \begin{cases}
													 	\frac{\theta}{1-\theta^2} &, k=1\\
													 	0 &, k>1
													 \end{cases}
	\end{equation*}
	Nótese que los requisitos de estacioneriedad (media y varianza constante+autocorrelación que depende del desplazamiento) se cumplen para cualquier MA independiendemente de sus parámetros.\\
	
	Si además $|\theta|<1$ se dice que el proceso $MA(1)$ es invertible en el sentido que puede ser representado por una serie que depende del pasado de la propia serie en vez del pasado de los errores
\end{frame}

%---------------------------------------------------
\subsubsection{Herramientas de identificación}
%---------------------------------------------------
\begin{frame}{Herramientas de identificación: Correlograma}
	\centering
	\begin{figure}
		\includegraphics[width = 0.85\linewidth]{fig/figure10.jpg}
	\end{figure}
\end{frame}
%---------------------------------------------------
\begin{frame}{Herramientas de identificación: Correlograma}
	\centering
	\begin{figure}
		\includegraphics[width = 0.85\linewidth]{fig/figure11.jpg}
	\end{figure}
\end{frame}

%---------------------------------------------------
\subsubsection{Promedios móviles autoregresivos (ARMA)}
%---------------------------------------------------
\begin{frame}{Promedios móviles autoregresivos (ARMA)}
	Es una combinación de un proceso AR y uno MA. La representación más sencilla es la de un ARMA(1,1)
		$$Z_t = \rho_1\breve{Z}_{t-1} +a_t +\theta a_{t-1} \textup{ con }  \breve{Z}_t = Z_t - \rho_0$$
	Como antes, para que el proceso sea estacionario se debe cumplir que $-1<\rho_1<1$ y para que sea invertible $-1<\theta<1$
\end{frame}
%%7) Teorema de Gauss-Markov ------------------------
%	%====================================================================================
\section{Elegir el modelo adecuado}
%====================================================================================
\begin{frame}{Elegir el modelo adecuado (1)}
	\begin{itemize}
		\item Los correlogramas deben darnos una primera impresión del proceso que esta detrás de cada serie.
		\item De ser necesario diferenciar la serie.
		\item De todos los candidatos elegir los más ``parsimoniosos''
		\item Por parsimonia se entiende aquellos que expliquen lo mismo usando menos (recuerde los términos calidad y cantidad)
	\end{itemize}
\end{frame}
%---------------------------------------------------
\begin{frame}{Elegir el modelo adecuado (2)}
	\begin{itemize}
		\item Criterios de parsimonía:
		\item \textcolor{red}Error cuadrático medio $= \sum \frac{e^2}{T-k}$
		\item Si hace memoria este es un componente del $R^2$ ajustado.
		\item Criterio de información de Akaike $= e^{\frac{2k}{T}}\sum \frac{e^2}{T}$
		\item Criterio de información de Scharz $=T\left( \frac{K}{T}\right)\sum \frac{e^2}{T}$
		\item En los tres casos lo que queremos es menor error (calidad) con el menor número de explicativas (cantidad)
		\item Por lo que a menor valor mejor es el modelo.
	\end{itemize}
\end{frame}
%%X) Referencias ------------------------------------
%	%===============================================================================
\section{Referencias}
%===============================================================================
\begin{frame}{Referencias}
	\begin{thebibliography}{00}
		\beamertemplatebookbibitems
		\bibitem{stwa}
		Stock and Watson (2011). Introduction to Econometrics.
		Third Edition; The Addison-Wesley Series in Economics.
		
		\beamertemplateonlinebibitems
		\bibitem{p-value}
		P-values. \href{http://www.tutor-homework.com/statistics_tables/statistics_tables.html}{\em Click aquí}
		
		\beamertemplatearticlebibitems
		\bibitem{tables}
		Tablas estadísticas. \href{http://verso.mat.uam.es/~pablo.fernandez/tablas_ProbI_2007-2008.pdf}{\em Click aquí}
		
		\beamertemplatearticlebibitems
		\bibitem{tables}
		Grados de Libertad. \href{https://en.wikipedia.org/wiki/Degrees_of_freedom_(statistics)}{\em Click aquí}
		
		\beamertemplateonlinebibitems
		\bibitem{p-value}
		Tests de significancia \href{http://www.stat.yale.edu/Courses/1997-98/101/sigtest.htm}{\em Click aquí}
		
		\beamertemplatebookbibitems
		\bibitem{catr}
		Cameron, C. and P. Trivedi (2011). Microeconometrics
		Using STATA. STATA press.
	\end{thebibliography}
\end{frame}







%1) Motivación para la regresión múltiple -----------
	%===============================================================================
\section{Definición del modelo de regresión simple}
%===============================================================================

%-------------------------------------------------------------------------------
\subsection{Terminología}
%-------------------------------------------------------------------------------
\begin{frame}{Terminología}
	\begin{description}
		\item[El modelo] $y=\beta_{0}+\beta_{1}x+u$
		\pause
		\item[$y$] Variable dependiente, variable explicada.
		\pause
		\item[$x$] Variable independiente, regresor, variable explicativa, variable de control o covariado.
		\pause
		\item[$u$] Término de error poblacional.
		\pause
		\item[$\beta_{0}$] Intercepto, constante o coeficiente no asociado con variables.
		\pause
		\item[$\beta_{1}$] Pendiente o coeficiente relacionado a $x$.
	\end{description}
\end{frame}
%---------------------------------------------------
\begin{frame}{Terminología}
	El MRLS es {\large $$ Y_{i} = \beta_{0} + \beta_{1}X_{i}+u_{i}$$}El subíndice $i$ va desde $i = 1$ a $i = n$ (última observación). $Y$ es la variable dependiente; $X$ es la variable dependiente; $\beta_{0}$ es el intercepto, constante o coeficiente no asociado con variables; $\beta_{1}$ es la pendiente o el coeficiente relacionado a $X_{i}$ y $u_{i}$ es el témino error. $Y$ y $X$ son datos.\\
	
	También, podemos denotar $\bar{Y}$ y $\bar{X}$ como el promedio muestral de las variables $Y$ y $X$ respectivamente. En término matriciales, tenemos
	$$ \underrightarrow{Y} = \left [ \begin{array}{c}
		50		\\
		30		\\
		20		\\
		\vdots	\\
		100
	\end{array}
	\right] 
	\quad \textup{y} \quad
	\underrightarrow{X} = \left [ \begin{array}{c}
		2		\\
		4		\\
		1		\\
		\vdots	\\
		6
	\end{array}
	\right]$$
\end{frame}

%2) Mecánica e interpretación de mínimos cuadrados
%	ordinarios --------------------------------------
	%===============================================================================
\section{Mecánica e interpretación de mínimos cuadrados ordinarios}
%===============================================================================

%-------------------------------------------------------------------------------
\subsection{Obtención de las estimaciones de MCO}
%-------------------------------------------------------------------------------
%---------------------------------------------------
\begin{frame}{Estimación (1)}
	Es la generalización del modelo de regresión simple:
	$$y=\beta_{0}+\beta_{1}x_1+\beta_{2}x_2+...+\beta_{k}x_k+u$$
	Donde:
	\begin{itemize}
		\item $y=X\widehat{\beta}+e$
		\item $\sum e_i^2=e'e$
		\item El criterio de optimización es el mismo sólo que ahora se tienen $k+1$ condiciones de primer orden:
		\bigskip
		\begin{itemize}
			\item $\sum e_i=e'1=0$
			\item $\sum e_i x_{1i}=e'x_1=0$
			\item $\sum e_i x_{2i}=e'x_2=0$, etc
		\end{itemize}
	\end{itemize}
\end{frame}
%---------------------------------------------------
\begin{frame}{Estimación (2)}
	Las $k+1$ ecuaciones pueden expresarse matricialmente de la siguiente forma:
	\begin{align*}
		e'X & = [0, 0, \ldots ,0] \\
		X'e & = [0, 0, \ldots ,0]' \\ 
		X'(y-X\widehat{\beta}) &= [0,0,...,0]' \\
		\widehat{\beta}&= (X'X)^{-1}X'y  \\
	\end{align*}
\end{frame}
%---------------------------------------------------
\begin{frame}{Estimación (3)}
	Alternativamente, cómo el objetivo es minimizar $\sum e_i^2=e'e=$
	\begin{align*}
		(y-X\beta)'(y-X\beta) & = y'y-y'X\beta-\beta'x'y+\beta'X'X\beta \\
		& = y'y-2y'X\beta+\beta'X'X\beta
	\end{align*}
	Condiciones de optimización:
	\bigskip
	\begin{description}
		\item[CPO:] $\frac{\partial e'e}{\partial\widehat{\beta}}=0-2X'y+2X'X\widehat{\beta}=0$
		\item[CSO:] $\frac{\partial^2 e'e}{\partial\widehat{\beta}^2}=2X'X$, (Definida positiva)
	\end{description}
\end{frame}

%-------------------------------------------------------------------------------
\subsection{Interpretación de la ecuación de regresión de MCO}
%-------------------------------------------------------------------------------
\begin{frame}{Interpretación}
	\begin{itemize}
		\item $\hat{y}=\hat{\beta_{0}}+\hat\beta_{1}x_{1}+\hat\beta_{2}x_{2}+...+\hat\beta_{k}x_{k}$\\
		Tomando el operador diferencial:
		\item $\Delta\hat{y}=\hat\beta_{1}\Delta x_{1}+\hat\beta_{2}\Delta x_{2}+...+\hat\beta_{k}\Delta x_{k}$ \\
		Así, manteniendo fijos $x_{2},...,x_{k}$, entonces
		\item $\Delta\hat{y}=\hat{\beta_{1}}\Delta x_{1}$\\
		Interpretar
		\bigskip
		\begin{enumerate}
			\item $Ln(y)=\hat\beta_{0}+\hat\beta_{1} Ln(x_{1})$
			\item $Ln(y)=\hat\beta_{0}+\hat\beta_{1} x_{1}$
			\item $y=\hat{\beta_{0}}+\hat\beta_{1} Ln(x_{1})$
		\end{enumerate}
	\end{itemize}
\end{frame}

%-------------------------------------------------------------------------------
\subsection{Sobre el significado de "mantener otros factores fijos" en regresión múltiple}
%-------------------------------------------------------------------------------

%-------------------------------------------------------------------------------
\subsection{Cambiar más de una variable independiente simultáneamente}
%-------------------------------------------------------------------------------

%-------------------------------------------------------------------------------
\subsection{Valores ajustados y residuales de MCO}
%-------------------------------------------------------------------------------
\begin{frame}{Valor ajustado o predicho}
	La línea de regresión poblacional es la relación que se mantiene entre $y$ y $x$ en promedio en la población.
	$$E(y|x_{1}, x_{2}) = \beta_{0}+\beta_{1}x_1+\beta_{2}x_2$$
	Los valores predichos $\widehat{y}$ y residuales $\widehat{uº}$ por MCO son
	\begin{align*}
		\hat{u} & = y - \hat{y}\\
		& \equiv y - (\hat{\beta}_{0}+\hat{\beta}_{1}x_{1}+\hat{\beta}_{2}x_{2})
	\end{align*}
\end{frame}

%-------------------------------------------------------------------------------
\subsection{Una interpretación "parcializada" de la regresión múltiple}
%-------------------------------------------------------------------------------
\begin{frame}{Recordando}
	\begin{enumerate}
		\item Si $y=\hat\beta_0+\hat\beta_1 x_1+\hat u=\hat y+\hat u$
		\begin{itemize}
			\item CPO:
			\item $\sum \hat u=0$
			\item $\sum \hat u x_1=0$
		\end{itemize}
		\item Dos ecuaciones, dos incógnitas. Resolviendo (Probar):
		\item $\hat{\beta}_1=\frac{\sum(x_i-\overline{x})(y_i-\overline y)}{\sum(x_i-\overline x)^2}=\frac{\sum(x_i-\overline x)y_i}{\sum(x_i-\overline x)^2}$
	\end{enumerate}
\end{frame}
%---------------------------------------------------
\begin{frame}{Generalizando}
	\begin{enumerate}
		\item Si $y=\hat\beta_0+\hat\beta_1 x_1+\hat\beta_2 x_2+\hat v=\hat y+\hat v$
		\begin{align}
			\sum \hat v & = 0 \\
			\sum \hat v x_1 & = 0 \\
			\sum \hat v x_2 & = 0
		\end{align}
		\item Imagina que:
		\begin{align}
			x_1 = \gamma_0+\gamma_1 x_2+\hat e=\hat{x}_1+\hat e
		\end{align}
		entonces CPO:
		\begin{itemize}
			\item $\sum \hat e=0$
			\item $\sum \hat e x_2=0$
		\end{itemize}
	\end{enumerate}
\end{frame}
%---------------------------------------------------
\begin{frame}{Estimación} De (2):
	\begin{align}
		\sum\hat v x_1 & = 0 \\
		\sum\hat v (\gamma_0+\gamma_1 x_2+\hat e) & = 0 \\
		\sum\hat v \hat e & = 0 \\
		\sum (y-\hat\beta_0-\hat\beta_1 x_1-\hat\beta_2 x_2) \hat e & = 0\\
		\hat{\beta}_1=\frac{\sum y \hat e}{\sum \hat{e}^2}
	\end{align}
\end{frame}

%-------------------------------------------------------------------------------
\subsection{Comparación de estimaciones de regresión simple y múltiple}
%-------------------------------------------------------------------------------
\begin{frame}{Condiciones de primer orden}
	$y_i=\widehat{\beta}_0+\widehat{\beta}_1 x_i+e_i$, resultado de
	minimizar la función $\sum e_i^2$ se tiene que las CPO son: 
		\begin{align*}
			\frac{\partial\sum e_i^2}{\partial \widehat{\beta}_0} &= \sum(y_i-\beta_0-\widehat{\beta}_1 x_i)(-1)=0 \\
			\frac{\partial\sum e_i^2}{\partial \widehat{\beta}_1} &= \sum(y_i-\widehat{\beta}_0-\beta_1 x_i)(-x_i)=0
		\end{align*}
	$(y-X\beta)'(y-X\beta)  = y'y-y'X\beta-\beta'x'y+\beta'X'X\beta= y'y-2y'X\beta+\beta'X'X\beta
	Condiciones de optimización:$
		\bigskip
		\begin{description}
			\item[CPO:] $\frac{\partial e'e}{\partial\widehat{\beta}}=0-2X'y+2X'X\widehat{\beta}=0$
			\item[CSO:] $\frac{\partial^2 e'e}{\partial\widehat{\beta}^2}=2X'X$, (Definida positiva)
		\end{description}
\end{frame}

%-------------------------------------------------------------------------------
\subsection{Bondad de ajuste}
%-------------------------------------------------------------------------------
\begin{frame}{Un paso más}
	\begin{align}
		Var(\beta_1|x) & = Var(\frac{\sum y \hat e}{\sum \hat{e}^2}|x) \\
		& = \frac{\sum \hat{e}^2 Var (y|x)}{(\sum \hat e^2)^2} \\
		& = \frac{\sigma^2}{\sum \hat{e}^2}
	\end{align}
	De la ecuación (4) se podría calcular el R cuadrado:\\
	$R^2_1=1-SCR_1/SCT_1=1-\sum \hat{e}^2/\sum (x_1-\overline{x})^2$;entonces\\
	$\sum \hat{e}^2=(1-R_1^2)\sum(x_1-\overline{x})^2$,
	entonces:\\
	\begin{align}
		&= \frac{\sigma^2}{SCT_1(1-R_1^2)}
	\end{align}
\end{frame}	
%---------------------------------------------------
\begin{frame}[fragile]{Ejemplo}
	Por favor, usando STATA ,descargue el conjunto de datos de crecimiento del sitio web de recursos de SW y realice una regresión basada en el siguiente modelo
		\begin{align*}
			Growth_i  = & \enskip  \beta_0 + \beta_1Tradeshare_i + \beta_2YearSchool_i\\
			& + \beta_3rev\_coups_i + \beta_4Assasinations_i \\
			& + \beta_5rgdp60_i
		\end{align*}
	La líneas de código son\\
		\begin{Stata code}{STATA code}
			{\tiny
				\texttt{\textcolor{codeblue}{use} \textcolor{codecrimson}{\textquotedblleft http://wps.aw.com/wps/media/objects/11422/11696965/empirical/empex\_tb/Growth.dta\textquotedblright}, clear}\\
				\texttt{\textcolor{codeblue}{browse}}\\
				\texttt{\textcolor{codeblue}{describe}}\\
				\texttt{\textcolor{codeblue}{reg} growth tradeShare yearsSchool rev\_coups assasinations rgdp60}}
		\end{Stata code}
\end{frame}

%-------------------------------------------------------------------------------
\subsection{Regresión por el origen}
%-------------------------------------------------------------------------------
%3) El valor esperado de los estimadores de MCO -----
	%===============================================================================
\section{Derivación de estimaciones de mínimos cuadrados ordinarios}
%===============================================================================

%-------------------------------------------------------------------------------	
\subsection{Valores ajustados y residuales}
%-------------------------------------------------------------------------------
\begin{frame}{Beta vs Beta \textit{sombrero}}
	\begin{itemize}
		\item $\beta_{0}$ y $\beta_{1}$ son los parámetros poblacionales (los verdadero parámetros del \textit{Proceso Generador de Datos o PGD})
		\item $\widehat{\beta}_{0}$ y $\widehat{\beta}_{1}$ son los estimadores de esos parámetros poblacionales ("Los \textit{betas} hechos por los mortales")
		\item Nos gustaría conocer los verdaderos parámetros (¡realmente lo haríamos!) Pero solo nosotros podemos hablar de estimaciones
		\item Recuerde que nos gustaría tener un estimador insesgado y eficiente.
	\end{itemize}
\end{frame}

%-------------------------------------------------------------------------------
\subsection{Propiedades algebraicas de las estadísticas MCO}
%-------------------------------------------------------------------------------
\begin{frame}{Propiedades algebraicas}
	\begin{itemize}
		\item $\sum_{i=1}^{n}\hat{u_{i}}=0 \Leftrightarrow \frac{\sum_{i=1}^{n}\hat{u_{i}}}{n}=0$
		\item $\sum_{i=1}^{n}x_{i}\hat{u_{i}}=0$
		\item $\overline{y}=\hat{\beta_{o}}+\hat{\beta_{1}}\overline{x}$
	\end{itemize}
\end{frame}
%------------------------------------------------
\begin{frame}{Suma Cuadrada}
	Cada observación se puede descomponer de una parte explicada y otra no explicada: $y{i}=\hat{y_{i}}+\hat{u_{i}}$. A partir de esto definimos:
	\begin{description}
		\item[STC] Suma total de cuadrados $\sum_{i=1}^{n}(y_{i}-\overline{y})^{2}$
		\item[SEC] Suma explicada al cuadrado $\sum_{i=1}^{n}(\hat{y_{i}}-\overline{y})^{2}$
		\item[SRC] Suma de residuos al cuadrado $\sum_{i=1}^{n}(\hat{u_{i}})^{2}$
		\item[Finalmente] \textcolor{red}{STC=SEC+SRC} Probar!
		\item[$R^{2}$] Es el indicador de ajuste más popular empleado para medir que tan bien el modelo se ajusta a los datos:\\
		$R^{2}=\frac{SEC}{STC}=1-\frac{SRC}{STC}$
	\end{description}
\end{frame}
%4) La varianza de los estimadores de MCO -----------
	%===============================================================================
\section{La varianza de los estimadores de MCO}
%===============================================================================
\begin{frame}{Varianza de los estimadores}
	\begin{itemize}
		\item \textcolor{red}{Suposición RLM5} Homocedasticidad: $Var(u/x_{1},x_{2},...,x_{k})=\sigma^{2} \Leftrightarrow Var(y/x)=\sigma^{2}$
		
		\item Los cinco supuestos hasta ahora mencionados son conocidos como los supuestos de
		\textcolor{red}{Gauss-Markov}
		
		\item $Var(\hat{\beta_{j}})=\frac{\sigma^{2}}{SC_{j}(1-R_{j}^{2})}$, donde
		\item $SC_{j}=\sum(x_{ij}-\overline{x_{j}})$ y $R_{j}^{2}$ es el $R^{2}$ de regresionar $x_{j}$ sobre todos los otros $x's$
	\end{itemize}
\end{frame}
%---------------------------------------------------
\begin{frame}{Varianza de los estimadores}
	\begin{itemize}
		\item Sin embargo, no conocemos $\sigma^{2}$ porque no observamos los errores poblacionales $u_{i}$
		\item Lo que se conoce son los residuos o errores muestrales de la estimación $\hat{u}_{i}$
		\item Entonces el $\sigma^{2}$ se estima a partir de los errores:\\ $\hat{\sigma}^{2}=\sum(\hat u_{i}^{2})/(n-k-1)=SRC/GL$
		\item GL (Grados de libertad) es el número de observaciones menos el número de parámetros estimados (constante más $k$ pendientes)
	\end{itemize}
\end{frame}

\subsection{Los componentes de las varianzas de MCO: multicolinealidad}
\subsection{Varianzas en modelos mal especificados}
\subsection{Estimación de s2: errores estándar de los estimadores de MCO}
%5) Eficiencia de OLS: el teorema de Gauss-Markov ---
	%===============================================================================
\section{Valores esperados y varianzas de los estimadores de MCO}
%===============================================================================

%-------------------------------------------------------------------------------
\subsection{Insesgadez del MCO}
%-------------------------------------------------------------------------------
\begin{frame}{Supuestos 1-4}
	\begin{enumerate}
		\item Linealidad de los parámetros
		$$y=\beta_{0}+\beta_{1}x+\mu$$
		\item Muestreo aleatorio
		\item Variación muestral de la variable explicativa
		\item Media condicional cero del error:
		$$E(\mu / x) = E(\mu) = 0$$
	\end{enumerate}
\end{frame}
%------------------------------------------------
\begin{frame}{Supuestos 1-4}
	Con los supuestos 1 a 4 se prueba que los estimadores son insesgados:
	\begin{align*}
		E(\hat{\beta}_{1}) &= \beta_{1}\\
		E(\hat{\beta}_{0}) &= \beta_{0}
	\end{align*}
\end{frame}
%------------------------------------------------
\begin{frame}{Estimadores insesgados}
	\begin{align*}
		\widehat{\beta}_1 &= \frac{\sum (x_{i}-\overline{x})(y_{i}-\overline{y})}{\sum (x_{i}-\overline{x})^2}\\
		& = \frac{\sum (x_{i}-\overline{x})\{\beta_1(x_{i}-\overline{x})+u_i\}}{\sum (x_{i}-\overline{x})^2}\\
		& = \beta_1\frac{\sum (x_{i}-\overline{x})^2}{\sum (x_{i}-\overline{x})^2} + \frac{\sum (x_{i}-\overline{x})u_i}{\sum (x_{i}-\overline{x})^2}\\
		& = \beta_1 + \frac{\sum (x_{i}-\overline{x})u_i}{\sum (x_{i}-\overline{x})^2} \tag{2}
	\end{align*}
	Tomando el esperado y asumiendo que $X$ no es estocástico (¡por cierto, una suposición fuerte!)
	\begin{align*}
		E[\widehat{\beta}_1] & = \beta_1 + E\left[\frac{\sum (x_{i}-\overline{x})u_i}{\sum (x_{i}-\overline{x})^2} \right]
	\end{align*}
\end{frame}
%------------------------------------------------
\begin{frame}{Estimadores insesgados}
	Finalmente
	\begin{align*}
		E[\widehat{\beta}_1] & = \beta_1 + E\left[\frac{\sum (x_{i}-\overline{x})u_i}{\sum (x_{i}-\overline{x})^2}\right]\\
		& = \beta_1 + \frac{\sum (x_{i}-\overline{x})E[u_i]}{\sum (x_{i}-\overline{x})^2}\\
		& = \beta_1
	\end{align*}
	Hemos demostrado que el estimador MCO es un estimador insesgado. Pero, todavía necesitamos la variación.
\end{frame}

%-------------------------------------------------------------------------------
\subsection{Varianza de lo estimadores MCO}
%-------------------------------------------------------------------------------
\begin{frame}{Supuestos 5: Homocedasticidad}
	Supuesto 5:
	$$Var(\mu / x) = \sigma^{2}$$
	Sabiendo que:
	$$\hat{\beta}_{1} = \beta_{1} + \frac{\sum (x_{i}-\overline{x})\mu_{i}}{\sum (x_{i}-\overline{x})^2}$$
	Se demuestra que:
	$$Var(\hat{\beta}_{1})=\frac{\sigma^{2}}{\sum (x_{i}-\overline{x})^2}$$
	$\sigma^{2}$ no es conocido, pero puede ser estimado a partir de los residuales $\epsilon$.
\end{frame}
%------------------------------------------------
\begin{frame}{Estimadores eficientes}
	Veamos detenidamente la expresión (2)
	$$\widehat{\beta}_1 = \beta_1 + \frac{\sum (x_{i}-\overline{x})u_i}{\sum (x_{i}-\overline{x})^2} \quad \rightarrow \quad \widehat{\beta}_1 - \beta_1 = \frac{\sum (x_{i}-\overline{x})u_i}{\sum (x_{i}-\overline{x})^2} $$
	llevando cuadrados a ambas expresiones
	$$\left[ \widehat{\beta}_1 - \beta_1\right]^2 = \left[\frac{\sum (x_{i}-\overline{x})u_i}{\sum (x_{i}-\overline{x})^2}\right]^2 $$
	tomando el valor esperado
	$$E\left[ \widehat{\beta}_1 - \beta_1\right]^2 = E\left[\frac{\sum (x_{i}-\overline{x})u_i}{\sum (x_{i}-\overline{x})^2}\right]^2 = \frac{\sum (x_{i}-\overline{x})^2E(u_i^2)}{\left[\sum (x_{i}-\overline{x})^2\right]^2}$$
	\textit{Nota:} $E\left[ \widehat{\beta}_1 - \beta_1\right]^2 \equiv \widehat{\sigma}_{\widehat{\beta}_1}^{2}$
\end{frame}
%------------------------------------------------
\begin{frame}{Estimadores eficientes}
	reescribiendo la expresión anterior y teniendo en cuenta la característica i.i.d de u
	\begin{align*}
		E\left[ \widehat{\beta}_1 - \beta_1\right]^2 & = \frac{\sum (x_{i}-\overline{x})^2E(u_i^2)}{\left[\sum (x_{i}-\overline{x})^2\right]^2}\\
		& = \frac{\sum (x_{i}-\overline{x})^2E[u_i^2]}{\left[\sum (x_{i}-\overline{x})^2\right]^2}\\
		& = E[u_i^2] \frac{\sum (x_{i}-\overline{x})^2}{\left[\sum (x_{i}-\overline{x})^2\right]^2} \tag{3}
	\end{align*}
	finalmente
	$$E\left[ \widehat{\beta}_1 - \beta_1\right]^2 = \sigma^{2}\frac{1}{\sum (x_{i}-\overline{x})^2}$$
	donde $E[u^2]=\sigma^{2} \equiv s_{u}^{2}$. Estamos asumiendo homocedasticidad.
\end{frame}
%------------------------------------------------
\begin{frame}{Estimadores eficientes}
	¡Casi ahí!. Solo lo que necesitamos saber es el error estándar del error (u). Sabemos que la desviación estándar de la estimación MCO -en el caso de un regresor- es
	\begin{align*}
		Var(\widehat{\beta}_1) = \sigma^{2}\frac{1}{\sum (x_{i}-\overline{x})^2} \tag{4}
	\end{align*}
	¿Qué es $\sigma^{2}$?. Necesitamos un estimador (insesgado) para $\sigma^{2}$. Entonces, tenemos la siguiente propuesta:
	$$\widehat{\sigma}^{2} = \frac{\sum u^2}{(n-2)}$$
	Entonces:
	$$Var(\widehat{\beta}_1) = \widehat{\sigma}^{2}\frac{1}{\sum (x_{i}-\overline{x})^2}$$
\end{frame}

%-------------------------------------------------------------------------------
\subsection{Estimador del error de la varianza}
%-------------------------------------------------------------------------------
\begin{frame}{Estimador de $\sigma^{2}$}
	\begin{align}
		\epsilon	& = y - \hat{y}\\
		& = (\beta_{o}+\beta_{1}x+\mu) - (\hat{\beta}_{0}+\hat{\beta}_{1}x)\\
		\epsilon	& = (\beta_{0} - \hat{\beta}_{0}) - (\hat{\beta}_{1} - \beta_{1}) + \mu\\
		\overline{\epsilon}	& = (\beta_{0} - \hat{\beta}_{0}) - (\hat{\beta}_{1} - \beta_{1}) + \overline{\mu},\enskip \textcolor{red}{promedio\enskip (3)}\\
		\epsilon - \overline{\epsilon}	& = (\mu - \overline{\mu}) - (\hat{\beta}_{1} - \beta_{1})(x - \overline{x}),\enskip \textcolor{red}{(3-4)} \\
		\epsilon	& = (\mu - \overline{\mu})- (\hat{\beta}_{1} - \beta_{1})(x - \overline{x}),\enskip \textcolor{red}{Ec. Normal}
	\end{align}
\end{frame}
%------------------------------------------------
\begin{frame}{Estimador de $\sigma^{2}$}
	Finalmente:
	$$\epsilon^{2} = (\mu - \overline{\mu})^{2} + (\hat{\beta}_{1} - \beta_{1})^{2}(x - \overline{x})^{2}-2(\mu - \overline{\mu})(\hat{\beta}_{1} - \beta_{1})(x - \overline{x})$$
	Aplicando sumatoria:
	\begin{align*}
		\sum\epsilon^2	& = \sum(\mu-\bar\mu)^2 \\
		&  +(\hat\beta_1 -\beta_1)^2\sum(x-\bar x)^2\\
		&  -2 (\mu-\bar\mu)(\hat\beta_1 -\beta_1)\sum(x-\bar x)
	\end{align*}
	Esperanza matemática:
	\begin{align*}
		E\left[\sum\epsilon^2\right]	&=\textcolor{red}{\sum E(\mu-\bar\mu)^2} \\
		&+ \textcolor{blue}{E(\hat\beta_1 -\beta_1)^2\sum(x-\bar x)^2}\\
		&-\textcolor{green}{2 E(\mu-\bar\mu)(\hat\beta_1 -\beta_1)\sum(x-\bar x)} 
	\end{align*}
\end{frame}
%------------------------------------------------
\begin{frame}{Estimador de $\sigma^{2}$}
	Se puede probar que:
	\begin{align*}
		\textcolor{red}{\sum E (\mu-\bar\mu)^2} & = (n-1)\sigma^2 \\
		\textcolor{blue}{E(\hat\beta_1 -\beta_1)^2\sum(x-\bar x)^2} & = \sigma^2 \\
		\textcolor{green}{2 E(\mu-\bar\mu)(\hat\beta_1 -\beta_1)\sum(x-\bar x)} & =  2\sigma^2 
	\end{align*}
	Finalmente:
	\begin{align*}
		E\left[\sum\epsilon^2\right] &= (n-1)\sigma^2 +  \sigma^2 -2\sigma^2  \\
		E\left[\sum\epsilon^2/(n-2) \right] &= \sigma^2 
	\end{align*}
	$\hat\sigma^2=\sum\epsilon^2/(n-2)$ es un estimador insesgado de $\sigma^2$.
\end{frame}

%6) Algunos comentarios sobre el lenguaje del análisis
%	de regresión múltiple ----------------------------
	%===============================================================================
\section{Regresión por el origen y Regresión sobre una constante}
%===============================================================================
\begin{frame}{Caso particular: Modelo ingenuo}
	Función a minimizar
	$$g(\alpha)=\sum_{i=1}^n \varepsilon _i ^2= \sum_{i=1}^n (Y_i-\alpha)^2$$
	CPO:
	$$\frac{\partial g(\alpha)}{\partial \alpha}=\sum_{i=1}^n \frac{\partial \varepsilon _i ^2}{\partial \alpha}=\sum_{i=1}^n \frac{\partial \varepsilon _i ^2}{\partial \varepsilon_i} \frac{\partial \varepsilon_i}{\partial \alpha}=0$$
	C2O:
	$$\frac{\partial^2 g(\alpha)}{\partial^2 \alpha}>0$$
\end{frame}
%X) Referencias ------------------------------------
	%===============================================================================
\section{Referencias}
%===============================================================================
\begin{frame}{Referencias}
	\begin{thebibliography}{00}
		\beamertemplatebookbibitems
		\bibitem{stwa}
		Stock and Watson (2011). Introduction to Econometrics.
		Third Edition; The Addison-Wesley Series in Economics.
		
		\beamertemplateonlinebibitems
		\bibitem{p-value}
		P-values. \href{http://www.tutor-homework.com/statistics_tables/statistics_tables.html}{\em Click aquí}
		
		\beamertemplatearticlebibitems
		\bibitem{tables}
		Tablas estadísticas. \href{http://verso.mat.uam.es/~pablo.fernandez/tablas_ProbI_2007-2008.pdf}{\em Click aquí}
		
		\beamertemplatearticlebibitems
		\bibitem{tables}
		Grados de Libertad. \href{https://en.wikipedia.org/wiki/Degrees_of_freedom_(statistics)}{\em Click aquí}
		
		\beamertemplateonlinebibitems
		\bibitem{p-value}
		Tests de significancia \href{http://www.stat.yale.edu/Courses/1997-98/101/sigtest.htm}{\em Click aquí}
		
		\beamertemplatebookbibitems
		\bibitem{catr}
		Cameron, C. and P. Trivedi (2011). Microeconometrics
		Using STATA. STATA press.
	\end{thebibliography}
\end{frame}






%---------------------------------------------------	
\note[itemize]{
	\item Agregar alguna nota
}
	
%------------------------------------------------------------------------------------
% End
%----
\begin{frame}
	\maketitle
\end{frame}
%------------------------------------------------------------------------------------
\end{document}		
%====================================================================================