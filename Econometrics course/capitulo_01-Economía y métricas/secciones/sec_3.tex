%===============================================================================
\section{La estructura de los datos económicos}
%===============================================================================

%-------------------------------------------------------------------------------
\subsection{Datos en serie de tiempo}
%-------------------------------------------------------------------------------
\begin{frame}{Datos en serie de tiempo}
	Es información recolectada en un instante en el tiempo a un conjunto de agentes que pueden ser personas, empresas, países. Ejemplos de este tipo
	de datos son las encuestas de opinión política, la ENAHO, variables macroeconómicas por países en un momento en el tiempo, etc. ``Muchas unidades de observación en un mismo momento en el tiempo''.
\end{frame}
%------------------------------------------------
\begin{frame}{Datos en serie de tiempo - Ejemplo}
	\centering
	{\small Desempleo en la zona euro}\\
	{\tiny (Ratios)} \\
	\smallskip
	\begin{tabular}{ ccc } 
		\hline
		Años & Desempleo & Desempleo juvenil \\
		\hline\hline
		2000 & 8.5 	& n.a  \\
		2001 & 8.1 	& n.a  \\
		2002 & 8.4 	& n.a  \\
		2003 & 8.8 	& n.a  \\
		2004 & 9.0 	& n.a  \\
		2005 & 9.1 	& n.a  \\
		2006 & 8.5 	& n.a  \\
		2007 & 7.6 	& n.a  \\
		2008 & 7.6 	& n.a  \\
		2009 & 9.6 	& 20.2 \\
		2010 & 10.1 & 20.9 \\
		2011 & 10.2 & 20.8 \\
		\hline
	\end{tabular}
\end{frame}

%-------------------------------------------------------------------------------
\subsection{Datos de corte transversal}
%-------------------------------------------------------------------------------
\begin{frame}{Datos de corte transversal}
	Este tipo de información es recolectada a un solo agente en un lapso de tiempo, como por ejemplo el PBI percápita y el gasto del gobierno de 1970 al 2007 de Perú, las ventas de la empresa X durante los últimos 20 años, etc. ``Una unidad de observación en varios momentos en el tiempo''.
\end{frame}
%------------------------------------------------
\begin{frame}{Datos de corte transversal - Ejemplo}
	\centering
	{\small Índice de Corrupción 2006} \\
	\smallskip
	\begin{tabular}{ ccc } 
		\hline
		Individuo & País & IPC puntaje \\
		\hline\hline
		1 & Finlandia & 9.6 \\
		2 & Islandia  & 9.6 \\
		3 & Dinamarca & 9.5 \\
		4 & Chile	  & 7.3 \\
		5 & USA		  & 7.3 \\
		6 & España	  & 6.8 \\
		7 & Perú	  & 3.3 \\
		8 & Ecuador	  & 2.3 \\
		9 & India	  & 3.3 \\
		\vdots & \vdots & \vdots \\
		\hline
	\end{tabular}
\end{frame}

%-------------------------------------------------------------------------------
\subsection{Datos panel}
%-------------------------------------------------------------------------------
\begin{frame}{Datos panel}
	\begin{description}
		\item[Pool data] Es así como se suele conocer a la unión de datos de corte transversal. En la práctica es el proceso de poner `una base sobre
		otra' y entre sus principales ventajas están las de incrementar el tamaño de la muestra y poder estudiar los efectos de determinado
		acontecimiento u política ocurrido en el momento intermedio del suceso: Encuesta en 1990 + Suceso + Encuesta en 1995
		\item[Datos de panel o longitudinales] En este caso las encuestas son aplicadas a los mismos individuos. Así se puede seguir el comportamiento en cuanto a ingresos
		y gastos de varias familias por dos o mas años o también la información macroeconómica de un conjunto de países por un determinado numero de años.
	\end{description}
\end{frame}
%------------------------------------------------
\begin{frame}{Datos panel - Ejemplo}
	\centering
	{\small Desempeño Económico} \\
	{\tiny (Ratios de crecimiento)}\\
	\smallskip
	\begin{tabular}{ cccc } 
		\hline
		Individuo & Año &País & PBI percápita \\
		\hline\hline
		1 & 1990 & USA & 2.32  \\
		1 & 1991 & USA & 0.12  \\
		1 & 1992 & USA & -2.04 \\
		2 & 1990 & ARG & -8.75 \\
		2 & 1991 & ARG & -3.66 \\
		2 & 1992 & ARG & 11.19 \\
		\vdots & \vdots & \vdots & \vdots \\
		2 & 1999 & ARG & -4.67 \\
		3 & 1991 & SPA & 3.56  \\
		\vdots & \vdots & \vdots & \\
		\hline
	\end{tabular}
\end{frame}