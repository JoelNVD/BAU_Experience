%===============================================================================
\section{Pasos en el análisis económico empírico}
%===============================================================================

%--------------------------------------------------------------------------------
\subsection{Experimentación}
%--------------------------------------------------------------------------------
\begin{frame}{Experimentación}
	\begin{itemize}[<+>]
		\item El camino más puro para medir el efecto causal es mediante un experimento.
		\item Técnicamente en una \textit{Prueba Controlada Aleatorizada} (RCT) se eligen dos grupos de manera aleatoria.
		\item Los \textit{tratados} reciben el \textit{tratamiento} y el resultado de la variable a analizar se compara con un grupo de \textit{control} .
		\item Experimentos son raros en econometría porque frecuentemente son antiéticos, difíciles de ejecutar satisfactoriamente o demasiados caros.
		\item La econometría ofrece un camino alternativo para aproximar el impacto.
	\end{itemize}
\end{frame}

%--------------------------------------------------------------------------------
\subsection{Datos experimentales y observacionales}
%--------------------------------------------------------------------------------
\begin{frame}{Datos experimentales y observacionales}
	\begin{itemize}[<+>]
		\item Los datos pueden provenir de dos fuentes: experimentos u observacionales.
		\item Los datos observacionales, se recopilan por medio de encuestas o registros administrativos.
		\item Cuando se tratan de experimentos estimar el impacto podría ser tan simple como restar promedios.
		\item Con los datos observacionales, el desafío de la econometría consiste en aislar el efecto del tratamiento de otros factores relevantes.
		\item En cualquier caso las bases de datos podrían ser clasificadas como \textit{corte transversal}, \textit{series de tiempo} y \textit{paneles de datos}.
	\end{itemize}
\end{frame}
