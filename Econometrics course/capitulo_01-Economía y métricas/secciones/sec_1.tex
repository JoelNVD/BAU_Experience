%===============================================================================
\section{¿Qué es Econometría}
%===============================================================================

%--------------------------------------------------------------------------------
\subsection{Econometría}
%--------------------------------------------------------------------------------
\begin{frame}{Econometría}
	\begin{itemize}
		\item ¿Econometría = ``medición económica''? \pause
		\item Literalmente sí, pero su alcance es más amplio.\pause
		\item Integra a: Estadística + Teoría económica + Matemática.\pause
	\end{itemize}
\end{frame}

%--------------------------------------------------------------------------------
\subsection{La caja negra}
%--------------------------------------------------------------------------------
\begin{frame}{La caja negra}
	\begin{itemize}
		\item Los economistas están interesado en entrar en la caja negra
			\begin{itemize}\pause
				\item ¿Cómo se generan los precios? ¿Cuál es el modelo detrás de los datos?\pause
				\item ¿Por qué algunos países son pobres? ¿Cuál es la fuente del crecimiento económico?\pause
				\item ¿En cuánto aumenta el salario por cada año de estudios?\pause
				\item ¿Cuánto será la inflación el próximo año? \pause
				\item ¿Cómo generan los estudiantes calificaciones? ¿Cuál es el determinante más importante para un desempeño exitoso?\pause
			\end{itemize}
		\item Algunas relaciones son completamente obvias, pero los economistas necesitamos "números"
	\end{itemize}
\end{frame}
%------------------------------------------------
\begin{frame}{La caja negra}
	\begin{itemize}[<+>]
		\item En todos los casos es posible encontrar al menos un modelo económico que relacione las variables.
		\item También es posible establecer modelos matemáticos a partir de lo anterior. 
		\item Estas preguntas tienen que ver con relaciones causales entre variables.
	\end{itemize}
\end{frame}