%====================================================================================
% Preamble
%------------------------------------------------------------------------------------
\documentclass{beamer}

% Essential Packages
\usepackage{lmodern}
\usepackage{booktabs}
\usepackage{tikz}
\usepackage{pgfplots}
\usepackage[accumulated]{beamerseminar}
\usepackage{graphicx}

% Apartadode Texto
\usepackage[utf8]{inputenc}
\usepackage[spanish]{babel}
\usepackage[T1]{fontenc}

% Apartado Matemático
\usepackage{amsmath}
\usepackage{amsfonts}
\usepackage{amssymb}
\usepackage{mathtools}

% Aparatado Tikz y pgfplots
\usepackage{tikz}
\usepackage{pgfplots}

\definecolor{orangeblack}{RGB}{228, 131, 10}
\definecolor{redblack}{RGB}{156, 75, 79}
\definecolor{redred}{RGB}{255, 51, 51}

% Apartado de íconos
\usepackage{fontawesome5}

% Apartado de Justificación del texto
\usepackage{ragged2e}
\justifying
\renewcommand{\raggedright}{\leftskip=0pt \rightskip=0pt plus 0cm}

% Apartado de colores
\usepackage{xcolor}

% Apartado Stata
\usepackage{listings}

\definecolor{codeblue}{rgb}{0.29296875, 0.51953125, 0.68359375}
\definecolor{codegreen}{rgb}{0.47265625, 0.62890625, 0.40234375}
\definecolor{codegray}{rgb}{0.95703125, 0.95703125, 0.95703125}
\definecolor{codecrimson}{rgb}{0.87109375,0.3984375,0.3984375}

\lstset{frame=tb,
	backgroundcolor=\color{codegray},
	aboveskip=3mm,
	belowskip=3mm,
	showstringspaces=false,
	columns=flexible,
	basicstyle={\ttfamily},
	numbers=left,
	numberstyle=\tiny\color{gray},
	keywordstyle=\color{codeblue},
	commentstyle=\color{codegreen},
	stringstyle=\color{codecrimson},
	breaklines=true,
	breakatwhitespace=true,
	tabsize=4,
	numbers=left,                    
	numbersep=5pt,
	frame=tlbr,framesep=2pt,framerule=0pt
}

% Apartado de block_colores
	% STATA code
	\newenvironment<>{Stata code}[1]{%
		\begin{actionenv}#2%
			\def\insertblocktitle{#1}%
			\par%
			\mode<presentation>{%
				\setbeamercolor{block title}{fg=black,bg=gray!15!white}
				\setbeamercolor{block body}{bg=codegray}
			}%
			\usebeamertemplate{block begin}}
		{\par\usebeamertemplate{block end}\end{actionenv}}	

% Apartado de configuración del Beamer
\mode<presentation> {
	\usetheme{Frankfurt}
	\setbeameroption{show notes}
	\setbeamercolor{item projected}{fg=white,bg=red}
	\setbeamertemplate{footline}[frame number]
	\usefonttheme[onlylarge]{structuresmallcapsserif}
	\usefonttheme[onlysmall]{structurebold}
	\usecolortheme{beaver}
	\setbeamercovered{transparent}
	\setbeamertemplate{navigation symbols}{}
}

	% Apartado transparencia de contenido
	\AtBeginSection[]
	{
		\begin{frame}<beamer>{Contenido}
			\tableofcontents[currentsection,currentsubsection]
		\end{frame}
	}
	\AtBeginSubsection[]
	{
		\begin{frame}<beamer>{Contenido}
			\tableofcontents[currentsection,currentsubsection]
		\end{frame}
	}
	
	% Apartado de logo
	\logo{\includegraphics[scale=.1]{figuras/logo-USAT.png}}

%====================================================================================
% Body
%====================================================================================
% Title Page
%-----------
\title[Capítulo 01]{Econometría Básica}
	\subtitle{Capítulo 01: Economía y Métricas}
	\author[José Valderrama \& Freddy Rojas]{José Valderrama \& Freddy Rojas \\
	\texttt{jtvalderrama@gmail.com \& frojasca@gmail.com} \faIcon{envelope} \\
		Universidad Católica Santo Toribio de Mogrovejo}
	\date[Septiembre de 2021]{Septiembre de 2021}
	
%------------------------------------------------------------------------------------
% Open
%---------
\begin{document}
	\rmfamily
		\begin{frame}
			\maketitle
		\end{frame}
%------------------------------------------------------------------------------------
% Before all
%-----------
	\begin{frame}
		\centering
		\vspace{2cm}
			\emph{\textit {``Ni teoría sin medida, ni medida sin teoría''.}} \\
		\vspace{2cm}
		\flushright
			Ragnar Frish en 1933. \\
			Primer numero de ECONOMETRICA.
	\end{frame}
%------------------------------------------------------------------------------------
% Sections
%---------
	\begin{frame}{Contenido}
		\tableofcontents
	\end{frame}

%1) ¿Qué es econometría? ---------------------------
	%===============================================================================
\section{Revisión de Probabilidades}
%===============================================================================

%-----------------------------------------------------------------
\subsection{Elementos de probabilidad}
%-----------------------------------------------------------------
\begin{frame}{Espacio muestral}
	\begin{itemize}
		\item Aleatoriedada. Es la falta de (cierta) predictibilidad sobre situaciones.
		\item El resultado es un resultado potencial mutuamente excluyente de un proceso aleatorio.
		\item Un (sub) espacio muestral (S) es la unión de todas las posibles realizaciones o resultados.
		\item ¿Qué es un evento (A)?. Un evento es una realización o un subconjunto de realizaciones que pertenecen al espacio muestral (S).
		\item La probabilidad de ocurrencia es la proporción de eventos verificables que provienen de un proceso aleatorio.
		\item Axiomas de la probabilidad:
		\begin{itemize}
			\item \textit{Axioma 1}: $0\leq P(A) \leq 1$
			\item \textit{Axioma 2}: $P(S)=1$
			\item \textit{Axioma 3}: $P(UA_{i}) = \sum P(A_{i})$
		\end{itemize}
	\end{itemize}
\end{frame}
%------------------------------------------------
\begin{frame}{Algunos ejemplos}
	Indicar lso eventos de los siguientes (sub) espacios (S):
	{\small
		\begin{itemize}
			\item $A=$ \{x : x es el tipo de carros que se estrellaron en invierno\}
			\item $B=$ \{x : x es el número de computadoras destruidas cada 15 días\}
			\item $C=$ \{x : x es la nacionalidad de las personas en el salón de clases\}
			\item $D=$ \{x : x es el número de errores tipográficos en un documento cada 5 páginas\}
			\item $E=$ \{x : x es el tiempo de reacción de Usain Bolt en todas las carreras \}
			\item $F=$ \{x : x es el estudiante que falla en la pregunta número 1 del cuestionario\}
			\item $G=$ \{x : es una carta que pertenece a una baraja estandar\}
	\end{itemize}}
\end{frame}
%------------------------------------------------
\begin{frame}{Variable aleatoria}
	Las variables aleatorias (VA) son continuas o discretas. Una VA discreta solo toma un conjunto discreto de valores, mientras que una VA continua toma un posioble valor continuo. Por ejemplo:
	\begin{itemize}
		\item El número de personas que están en el salón depués de empezar cada clase.
		\item Puntaje en un cuestionario.
		\item Grados Fahrenheit o Celsius.
		\item Tiempo.
	\end{itemize}
	Puedes tranformar o convertir VA continuas en VA discretas fácilmente.
\end{frame}
%-----------------------------------------------------------------
\subsection{fdp y fda}
%-----------------------------------------------------------------
{\small
	\begin{frame}{Distribución probabilística y acumulada de una VA Discreta}
		La distribución de probabilidad es la probabilidad de que ocurra cada resultado o evento. Por ejemplo, observamos la cantidad de fallas de computadoras el primer día de clase; con base en una muestra de 10 estudiantes podemos construir la siguiente tabla:
		\begin{table}[!htbp]
			\centering
			\begin{tabular}{*7c}
				\hline
				{} &  \multicolumn{6}{c}{Resultados} \\
				{} & 0 & 1 & 2 & 3 & 4 & $\geq$5\\
				\hline
				Número  	 & 6 	& 0    & 3    & 0    & 1    & 0     \\
				fdp muestral & 0.60 & 0.00 & 0.30 & 0.00 & 0.10 & 0.000 \\
				fda muestral & 0.60 & 0.60 & 0.90 & 0.90 & 1.00 & 1.000 \\
				\hline
			\end{tabular}
		\end{table}
		formalmente, calculamos las probabilidades de la siguiente manera
		\begin{align*}
			&\textup{muestra base\, fdp}\enskip (\textup{fallas}=j) = \frac{(\#\textup{fallas}=j)}{n}\\
			&\textup{muestra base\, fda}\enskip (\textup{fallas}=j) = \sum_{i=0}^{j} 	\frac{(\#\textup{fallas}=j)}{n}
		\end{align*}
\end{frame}}
%------------------------------------------------
\begin{frame}{Histograma en diferentes tamaños de muestra}
	\begin{figure}
		\centering
		\includegraphics[scale=.40]{figuras/histograma_discr.png}
	\end{figure}
\end{frame}
%------------------------------------------------
{\small
	\begin{frame}{Distribución Bernoulli}
		Una distribución donde el resultado toma solo 2 valores: $0$ y $1$ es la distribución de Bernoulli. Es un caso particular de Distribución Binomial. Por ejemplo; Sea $G$ el género de la siguiente persona que ingrese al aula después de que comience la clase, donde $G=0$ indica que la persona es hombre y $G=1$ si esa persona es mujer. Los resultados de $G$ y sus probabilidades son:
		\begin{equation*}
			G =	\begin{cases}
				1\enskip \textup{con probabilidad}\enskip p\\
				0\enskip \textup{con probabilidad}\enskip 1-p
			\end{cases}
		\end{equation*}
		Además, podemos configurar un espacio con más de $2$ resultados pero finito. Por ejemplo, lanzar $2$ dados: $M$ es la suma de puntos en $2$ dados después de lanzarlos. Por lo tanto, $p(M=1)=0$
		\begin{gather*}
			p(M=2)= \frac{1}{36};p(M=3)= \frac{1}{36}; \ldots;p(M=7)= \frac{1}{36};\\
			\ldots \enskip \textup{y}\enskip p(M=12)= \frac{1}{36}
		\end{gather*}
\end{frame}}
%------------------------------------------------
\begin{frame}{Distribución probabilística y acumulada de una VA Continua}
	Por ejemplo, asumimos que las calificaciones se distribuyen normalmente con media y varianza iguales a 3 y 2 respectivamente.
	\begin{table}
		\centering
		\begin{tabular}{cccc}
			\hline
			Intervalo & Freq. Abs. & Freq. Rel. & Freq. Acum.\\
			\hline
			$[0;2[$  & 1.0 & 0.10 & 0.10\\
			$[2;4[$  & 4.0 & 0.40 & 0.50\\
			$[4;6[$  & 4.0 & 0.40 & 0.90\\
			$[6;8[$  & 1.0 & 0.10 & 1.00\\
			$[8;10]$ & 0.0 & 0.00 & 1.00\\
			\hline
		\end{tabular}
	\end{table}
\end{frame}
%------------------------------------------------
\begin{frame}{El histograma en diferentes tamaños de muestra.}
	\begin{figure}
		\centering
		\includegraphics[scale=.40]{figuras/histograma_conti.png}
	\end{figure}
\end{frame}

%-----------------------------------------------------------------
\subsection{Valor esperado, media muestral y varianza}
%-----------------------------------------------------------------
\begin{frame}{Valor esperado, media muestral y varianza}
	Tenemos las siguientes definiciones y/o notaciones
	\begin{itemize}
		\item Variable aleatoria. Denotamos las variables aleatorias como $Y$.
		\item Una realización $i$ de $Y$. Es un resultado o valor observado. Denotamos eso como $y_{i}$.
		\item Valor esperado o media. Es el valor promedio a largo plazo de la variable aleatoria. Empíricamente, el valor esperado se puede aproximar como la media de la muestra cuando el número de observaciones o el tamaño de la muestra llega al infinito (o alcanza un número grande). Denotamos el valor esperado de una variable aleatoria como $E(Y)$ o $\mu_{Y}$.
		\item Media o promedio muestral. Es el valor promedio de la variable aleatoria. Denotamos la media de una variable aleatoria como $\mu_{Y, N}$.
		\item Diferencia. Es una medida de dispersión de la variable aleatoria. Denotamos la varianza como $\sigma_{Y}^{2}$.
	\end{itemize}
\end{frame}
%------------------------------------------------
\begin{frame}{Valor esperado, media muestral y varianza}
	Tome en cuenta que
	\begin{itemize}
		\item El valor esperada ($E$)  es un operador lineal.
		$$Eg(X) \geq g(E(X))$$
		\item Algunas reglas con constantes y variables aleatorias
		$$E(a+b) = a+b$$
		$$E(aX+b)=aE(X)+b$$
		$$E(XY)=E(X)E(Y) \textup{ si ambas variables son independientes}$$
	\end{itemize}
\end{frame}
%------------------------------------------------
\begin{frame}{Valor esperado (media), media muestral y varianza}
	Supongamos que tenemos las realizaciones para la variable $Y$. Por lo tanto,  tenemos $y_{1},\ y_{2},\ y_{3}, \ldots y_{k}$. También, conocemos las pobabilidad asociadas; estas son $p_{1},\ p_{2},\ p_{3}, \ldots 1 - \sum_{i}^{k-1}p_{y}$. Entonces, el valor esperado de $Y$, denotado por $E(Y)$ es $$E(Y) = p_{y_{1}}y_{1}+p_{y_{2}}y_{2}+p_{y_{3}}y_{3}+\ldots+p_{y_{k}}y_{k} = \sum_{i=1}^{k}p_{y_{i}}y_{i}$$ La suma anterior cubre todo el espacio de Y. En el caso de variables aleatorias continuas $$E(Y) = \int \limits_{\underline{y}}^{\bar{y}}yf(y)dy$$
\end{frame}
%------------------------------------------------
\begin{frame}{Desviación estandar y Varianza}
	La varianza de la variable aleatoria discreta $Y$ es $$\sigma_{Y}^{2} \equiv var(Y) = E[Y-\mu_{Y}]^2 = \sum_{i=1}^{k}(y_{i}-\mu_{k})^2p_{i}$$ En el caso de variables aleatorias continuas, la varianza es $$\sigma_{Y}^{2} \equiv var(Y) = E[Y-\mu_{Y}]^2 = \int \limits_{\underline{y}}^{\bar{y}}yf(y)dy$$ La desviación estandar de $Y$ es $\sigma_{Y}$ y es la raiz cuadrada de la varianza. ¿Cuál es la covarianza de las variables discretas y continuas?
\end{frame}
%------------------------------------------------
\begin{frame}{Ejercicio}
	Por ejemplo
	\begin{itemize}
		\item $G \thicksim B(1,p) \equiv B(p)$, entonces la media es $$E(G) = 0 \cdot  P(G=0) + 1 \cdot P(G=1)=p$$ y la varianza
		\begin{align*}
			\textup{var}(G) & \equiv E(G^2)-\mu_{G}^{2}\\
			& = 0^2 \cdot P(G=0)+1^2\cdot P(G=1)-p^2\\
			& = p(1-p)
		\end{align*}
		\item $X=5+G$, donde $G \thicksim B(p)$; entonces\\
		$E(X)=5\cdot P(G=0)+6P\cdot P(G=1)=5+p$;\\
		var$(X)= p(1-p)$
	\end{itemize}
\end{frame}
%------------------------------------------------
\begin{frame}
	\begin{itemize}
		\item $Y \thicksim N(0,1)$, entonces la media es $$E(Y)=0$$ y la varianza
		\begin{align*}
			\textup{var}(Y) & \equiv E(Y^2) - \mu_{Y}^{2}\\
			& = 1 
		\end{align*}
		\item $Z \thicksim N(5,2)$; entonces $E(Z)=5$; var$(Z)=2$
		\item $\frac{Z-5}{\sqrt{2}}\thicksim N(0,1)$
	\end{itemize}
\end{frame}
%------------------------------------------------
\begin{frame}{otros momentos}
	La curtosis es un indicador de cuánta masa hay en las colas de la distribución. El punto de referencia para este indicador es la distribución normal; en este caso la curtosis es igual a 3. La curtosis se calcula de la siguiente manera:
	$$\textup{Kurtosis} = E\left( \frac{Y-E(Y)}{\sigma_{Y}}\right)^4$$
	La asimetría es un indicador de la simetría de la distribución. Un valor positivo (negativo) indica que la distribución tiene una cola larga a la derecha (izquierda). El punto de referencia para este indicador es la distribución normal; en este caso, la asimetría es igual a 0. La asimetría se calcula de la siguiente manera
	$$\textup{Skewness} = E\left( \frac{Y-E(Y)}{\sigma_{Y}}\right)^3$$
\end{frame}
%------------------------------------------------
\begin{frame}{Una advertencia importante: valores esperados versus media muestral}
	Intente el siguiente ejercicio; descargue este conjunto de datos aquí. Luego, calcule las estadísticas de media, error estándar y covarianza para las ... primeras 5, 10, 50, 100 y 500 observaciones de la variable llamada rer y complete la siguiente tabla:\\
	\bigskip
	{\small
		\centering
		Estadísticas de muestra\\
		\smallskip
		\begin{tabular}{ cccc } 
			\hline
			Tamaño de muestra & Promedio & Error estandar & Cov con inf \\
			\hline
			5   & & & \\
			10  & & & \\
			50  & & & \\
			100 & & & \\
			50  & & & \\
			\hline
			\\                                                                                                                   
		\end{tabular}\\}
	Tener en cuenta que $E(rer) = 3$; var(rer)=2 y\\
	cov(rer,inf)=0.34
\end{frame}
%------------------------------------------------
\begin{frame}{Expresiones claves}
	Parámetros; $a, b, c \ldots;$ variables aleatorias $X, Y$
		\begin{align*}
			E(a+bX+cY) &=a+bE(X)+cE(Y)\\
			&=a+b\mu_{X}+c\mu_{Y}
		\end{align*}
		\begin{align*}
			Var(a+bX+cY) &=a^2Var(X)+b^2Var(Y)+2abCov(X,Y)\\
			&=a^2\sigma_{X}^{2}+b^2\sigma_{Y}^{2}+2ab\sigma_{XY}
		\end{align*}
		\begin{align*}
			Var(Y) & \equiv \sigma_{Y}^{2} = E(Y-E(Y))^2\\
			& = E(Y^2)-2E(Y)E(Y)+[E(Y)]^2\\
			& = E(Y^2) - [E(Y)]^2\\
			& = E(Y^2) - \mu_{Y}^{2}
		\end{align*}
\end{frame}
%------------------------------------------------
\begin{frame}{Expresiones claves}
	Parámetros; $a, b, c \ldots;$ variables aleatorias $X, Y$
		\begin{align*}
			\textup{cov}(X,Y) & \equiv E([X-E(X)][Y-E(Y)])\\
			& E(XY - E(X)Y-XE(Y)+E(X)E(Y))\\
			& E(XY-\mu_{x}Y-X\mu_{Y}+\mu_{X}\mu_{Y})\\
			& E(XY)-2\mu_{X}\mu_{Y}+\mu_{X}\mu_{Y}\\
			& E(XY)-\mu_{X}\mu_{Y}\\
			& \sigma_{XY}
		\end{align*}
	aquí $E(XY)$ se denomina covarianza bruta o no centrada entre $X$ e $Y$
\end{frame}

%-----------------------------------------------------------------
\subsection{Distribuciones conjuntas, marginales y condicionales}
%-----------------------------------------------------------------
\begin{frame}{Distribución conjunta}
	La distribución conjunta involucra al menos 2 variables aleatorias y es la probabilidad de observar resultados (asociados a estas variables aleatorias) simultáneamente.\\
	\smallskip
	{\small
		\centering
		Distribución conjunta\\
		\textcolor{red}{(en rojo)}\\
		\smallskip
		\begin{tabular}{c|cccc|c} 
			& \multicolumn{4}{c|}{Y} & \\
			X & 2 & 3 & 4 & 5 & Total \\
			\hline
			0 & \textcolor{red}{0.0} & \textcolor{red}{0.0}& \textcolor{red}{0.08} & \textcolor{red}{0.40} & 0.48 \\
			1 & \textcolor{red}{0.2}& \textcolor{red}{0.2}& \textcolor{red}{0.10} & \textcolor{red}{0.02} & 0.52 \\
			\hline
			Total & 0.2 & 0.2 & 0.18 & 0.42 & 1.00
		\end{tabular}\\}
	\medskip
	Según esta distribución la probabilidad de observar 5 fiestas asistidas y aprobar el curso de econometría $P (y = 5, x = 1)$ es del 2\%. Todos estos 8 eventos posibles son mutuamente excluyentes y constituyen el espacio muestral; las 8 probabilidades suman 1.
\end{frame}
%------------------------------------------------
\begin{frame}{Distribución marginal}
	La distribución marginal de una variable aleatoria es la distribución de probabilidad en sí. Esa es la probabilidad de observar resultados (asociada a esta variable aleatoria).\\
	\smallskip
	{\small
		\centering
		Distribución marginal\\
		\textcolor{blue}{(en azul)}\\
		\smallskip
		\begin{tabular}{c|cccc|c} 
			& \multicolumn{4}{c|}{Y} & \\
			X & 2 & 3 & 4 & 5 & Total \\
			\hline
			0 & 0.0 & 0.0& 0.08 & 0.40 & \textcolor{blue}{0.48} \\
			1 & 0.2& 0.2& 0.10 & 0.02 & \textcolor{blue}{0.52} \\
			\hline
			Total & \textcolor{blue}{0.2}& \textcolor{blue}{0.2}& \textcolor{blue}{0.18} & \textcolor{blue}{0.42} & 1.00
		\end{tabular}\\}
	\medskip
	Según esta distribución, la probabilidad de observar 5 fiestas $P (y = 5)$ es del 42\%. Los 4 posibles resultados de $Y$ son mutuamente excluyentes y constituyen un espacio muestral para $Y$; las 4 probabilidades suman 1.
\end{frame}
%------------------------------------------------
{\small
	\begin{frame}{Distribución condicional}
		La distribución condicional de una variable aleatoria es la distribución de probabilidad dada la realización de otra variable aleatoria. Formalmente, es la probabilidad de algún evento cuando el espacio muestral está restringido a otro evento.\\
		\smallskip
		{\centering
			Distribución condicional\\
			\textcolor{purple}{(en rojo)}\\
			\smallskip
			\begin{tabular}{c|cccc|c} 
				& \multicolumn{4}{c|}{Y} & \\
				X & 2 & 3 & 4 & 5 & Total \\
				\hline
				0 & \textcolor{purple}{0.0/0.2} & \textcolor{purple}{0.0/0.2}& \textcolor{purple}{0.08/0.18} & \textcolor{purple}{0.40/0.42} & n.d \\
				1 & \textcolor{purple}{0.2/0.2}& \textcolor{purple}{0.2/0.2}& \textcolor{purple}{0.10/0.18} & \textcolor{purple}{0.02/0.42} & n.d \\
				\hline
				Total & 1 & 1 & 1 & 1 & n.d
			\end{tabular}\\}
		
		\medskip
		Según esta distribución, la probabilidad condicional de observar un éxito en el curso (es decir, pasar la maldición) dado que asistieron 5 fiestas $P (x = 1 | y = 5)$ es 4.7\%. Todos los pares de probabilidades dados Y son mutuamente excluyentes y constituyen el espacio muestral dado un valor particular para Y; las 2 probabilidades suman 1 en cada caso.
\end{frame}}
%------------------------------------------------
\begin{frame}{Ejercicio}
	Calculemos la probabilidad marginal y condicional (dad la información de probabilidad conjunta) mediantes las siquientes expresiones
	$$P(Y=y) = \sum_{i=1}^{k}P(X=x, Y=Y) \quad (\textup{Pro. Marginal})$$
	$$P(Y=y|X=x) = \frac{P(X=x,Y=y)}{P(X=x)} \quad (\textup{Pro. Condicional})$$
	La última expresión es una aplicación del teorema de Bayes (es decir,$ P (A, B) = P (A | B) P (B)$)
\end{frame}
%------------------------------------------------
\begin{frame}{Expresiones claves adicionales}
	\begin{itemize}
		\item Esperanza condicional
		\begin{align*}
			E(Y|X=x) &= \sum_{i=1}^{I} y_{i} P(Y=y_{i}|X=x)\\
			&= \sum_{i=1}^{I}y_{i}\frac{P(X=x,Y=y_{i})}{P(X=x)}
		\end{align*}
		\item La ley de las expectativas iteradas (LEI)
		\begin{align*}
			E(Y) &= \underset{X}{E}[E(Y|X)]\\
			&= \sum_{i=1}^{k}E(Y|X=x_{i})P(X=x_{i})
		\end{align*}
	\end{itemize}
\end{frame}
%------------------------------------------------
\begin{frame}{Expresiones claves adicionales}
	\begin{itemize}
		\item Varianza condicional
			\begin{align*}
				\textup{var}(Y|X=x) = \sum_{i=1}^{I}[y_{i}-E(Y|X=x)]^2P(Y=y_{i}|X=x)
			\end{align*}
		\item Independencia
			\begin{align*}
				P(Y=y,X=x)=P(Y=y)P(X=x)
			\end{align*}
		\item Covarianza
			\begin{align*}
				\textup{cov}(Y,X)=\sum_{j=1}^{k}\sum_{i=1}^{I}[y_i-\mu_{Y}][x_j-\mu_{X}]P(Y=y_i,X=x_j)
			\end{align*}
	\end{itemize}
\end{frame}
%------------------------------------------------
\begin{frame}{Expresiones claves adicionales}
	\begin{itemize}
		\item Correlación (Parámetro de Pearson)
		$$\textup{corr}(X,Y)=\frac{\textup{cov}(X,Y)}{\sqrt{\textup{var}(Y)\textup{var}(X)}}$$
		La correlación mide la asociación estadística entre 2 variables (aleatorias). Se dice que las variables aleatorias no están correlacionadas si $\textup{corr}(X,Y)=0$. Es important mencionar que la correlación siempre está entre -1 y 1.
	\end{itemize} Además, debe tener en cuenta lo siguiente
	\begin{itemize}
		\item Si $E(Y|X) = \mu_{Y}$ (Independencia); entonces $\textup{cov}(X,Y)=0 \rightarrow \textup{corr}(X,Y) = 0$. La independencia no implica correlación, pero lo contrario no se cumple.
		\item $|\sigma_{X,Y}| \leq \sqrt{\sigma_{X}^{2}}\sqrt{\sigma_{Y}^{2}}$
		\item cov$(a+bX+cV,Y) = b\sigma_{X,Y} + c\sigma_{V,Y}$
	\end{itemize}
\end{frame}
%-----------------------------------------------------------------
\subsection{Distribuciones}
\begin{frame}{Distribución normal}
	El siguiente panel de gráficos muestra el fdp y el fda normal estándar.
		\begin{figure}
			\centering
			\includegraphics[scale=.22]{figuras/dist1.png}
		\end{figure}
	La distribución normal estandar se denota por $N(0,1)$.\\
	La fdp y fda se notan como $\phi(X)$ y $\Phi(X)$ respectivamente.
\end{frame}
%------------------------------------------------
\begin{frame}{Distribución Chi-Cuadrado}
	El siguiente panel de gráficos muestra la fdp y la fda de una Chi-cuadrado bajo diferentes grados de libertad.
		\begin{figure}
			\centering
			\includegraphics[scale=.22]{figuras/dist2.png}
		\end{figure}
	los gráficos de esta distribución son denotadas como $\chi^{2}(g.l.)$ o $\chi_{(g.l.)}^{2}$
\end{frame}
%------------------------------------------------
\begin{frame}{Distribución t-student}
	El siguiente panel de gráficos muestra la fdp y la fda de una t-student bajo diferentes grados de libertad.
		\begin{figure}
			\centering
			\includegraphics[scale=.22]{figuras/dist3.png}
		\end{figure}
	los gráficos de esta distribución son denotadas como $t_{g.l.}$
\end{frame}
%------------------------------------------------
\begin{frame}{Distribución F}
	El siguiente panel de gráficos muestra la fdp y la fda de una F bajo diferentes grados de libertad.
		\begin{figure}
			\centering
			\includegraphics[scale=.22]{figuras/dist4.png}
		\end{figure}
	los gráficos de esta distribución son denotadas como $F(g.l. 1 , g.l. 2)$
\end{frame}
%------------------------------------------------
\begin{frame}{Relación entre funciones}
	\begin{figure}
		\centering
		\includegraphics[scale=.30]{figuras/funciones.png}
	\end{figure}
\end{frame}
%------------------------------------------------
\begin{frame}{¿Sabes cómo leer tablas estadísticas?}
	\begin{description}
		\item[Prob 1] En la Prueba 1, los estudiantes no informaron los valores de las tablas estadísticas. Les pido que calculen los valores de $F (2, 26)$, $\chi^2(60)$ con un nivel de significancia de 0.05; y $t_{60}$ con un nivel de significancia de 0.005 (considere dos colas para eso). [\textit{Sugerencia: un nivel de significancia = área bajo la curva comenzando desde la izquierda}]
			\begin{align*}
				&\textup{Respuestas}\\
				&F(2,6)=3.37\\
				&\chi^{2}(60)=79.08\\
				&t_{60}=2.66
			\end{align*}
		\item[Prob 2] x es una variable aleatoria normal  estándar, calcule lo siguiente $a) P (x> 0)$, $b) P (x> 2.99)$ y $c) P (1.96> x> 1.96)$
	\end{description}
\end{frame}

%-----------------------------------------------------------------
\subsection{Aproximaciones de muestras grandes}
%-----------------------------------------------------------------
{\small
\begin{frame}{Propiedades de la Ley de Grandes Números y Consistencia}
	\begin{itemize}
		\item La ley de los grandes números (LGN) o la ley de los promedios
		\item Debido a que tenemos un límite de observaciones o una pequeña muestra observable, necesitamos aproximar las distribuciones utilizando una basada en muestras grandes (distribución asintótica).
		\item Para usar una aproximación para distribuciones (en este caso la normal) necesitamos (al menos) estar seguros de que la media se conoce en muestras grandes. Para ello nos basamos en el LGN y el teorema del límite central.
		\item La LGN dice que cuando el tamaño de la muestra es grande, el promedio muestral ($\overline{Y}$) estará más cerca a $\mu_{Y}$ con una alta probabilidad. Denotamos esta propiedad como $\overline{Y} \xrightarrow{p} \mu_{Y}$.
		\item Formalmente, si $Y$ son independiente e idénticamente distribuidas con $E(Y)=\mu_{Y}$ y si los valores atípicos son poco probables, entonces $\overline{Y} \xrightarrow{p} \mu_{Y}$.
	\end{itemize}
\end{frame}
}
%------------------------------------------------
\begin{frame}{Propiedades de la Ley de Grandes Números y Consistencia}
	En el caso de que $X \thicksim \chi^{2}(2)$; sabemos que $E[\chi^{2}(2)]=\mu_{\chi^{2}}=2$ y var$[\chi^{2}(2)]$ = 4 $< \infty$. Por lo tanto tenemos que $\overline{X} \xrightarrow{p} \mu_{X}$.
		\begin{figure}
			\centering
			\includegraphics[scale=.22]{figuras/lgn.png}
		\end{figure}
\end{frame}
%------------------------------------------------
\begin{frame}{Teorema del Límite Central}
	\begin{itemize}
		\item Ahora, sabemos que esta nueva variable $\overline{Y}$ tiene una media igual a $\mu_{Y}$ en muestras grandes. ¿Cuál es la forma de la distribución límite (o asintótica)?
		\item El teorema del límite central (CLT) dice que la distribución de una media muestral ($\overline{Y}$) está bien aproximada por una distribución normal cuando $n$ es grande.
		\item Recuerde que la media es $\mu_{Y}$ y la varianza $\frac{\sigma_{Y}^{2}}{n}$ para $Y$.
		\item La distribución asintótica de $\overline{Y}$ no depende de la distribución de $Y$.
		\item Esta simplificación (a punto de utilizar distribuciones normales como asintóticas) subyace a la teoría de la regresión.
	\end{itemize}
\end{frame}
%------------------------------------------------
\begin{frame}{Teorema del Límite Central}
	En el caso de $X \thicksim U(0,1)$; sabemos que $E[X]=0.5$ y var$[X]=\frac{1}{12}$. Por el TLC $\overline{X} \xrightarrow{d} N(0.5, \sigma_{\overline{X}}^{2})$.
		\begin{figure}
			\centering
			\includegraphics[scale=.22]{figuras/tlc.png}
		\end{figure}
\end{frame}

%-----------------------------------------------------------------
\subsection{Conclusiones}
%-----------------------------------------------------------------
\begin{frame}{Conclusiones}
	\begin{itemize}
		\item Recuerde que el valor esperado, cuando existe, es casi con seguridad el límite de la media muestral a medida que el tamaño de la muestra crece hasta el infinito. Por tanto, el valor esperado puede entenderse por la ley de los grandes números
		\item La expectativa es un operador lineal.
		\item La independencia no implica correlación, pero lo contrario no es válido.
		\item La media, la desviación estándar, la asimetría y la curtosis son momentos de las distribuciones.
		\item LLN dice que cuando el tamaño de la muestra es grande, la media de la muestra $(\overline{Y})$ estará cerca de $\mu_{Y}$ con una probabilidad muy alta
		\item La distribución asintótica de $\overline{Y}$ no depende de la distribución de $Y$.
	\end{itemize}
\end{frame}
%2) Pasos en la análisis económico empírico --------
	%===============================================================================
\section{Revisión de Estadística}
%===============================================================================

%-----------------------------------------------------------------
\subsection{Estimadores y propiedades}
%-----------------------------------------------------------------
\begin{frame}{Estimadores}
	\begin{itemize}
		\item Un estimador es la aproximación a parámetros o momentos poblacionales.
		\item Un estimador es una función de datos.
		\item Denotamos un estimador como un "sombrero".
		\item ¿Cómo elegimos un estimador? Debemos ocuparnos de lo siguiente
			\begin{itemize}
				\item Nos gustaría tener la siguiente propiedad: imparcialidad (es decir, $E (\hat{y}) = \mu_{Y})$
				\item Otra propiedad deseable: eficiencia (la más pequeña
				diferencia).
			\end{itemize}
	\end{itemize}
\end{frame}
%------------------------------------------------
\begin{frame}{Propiedades}
	Nos gustaría tener las siguientes propiedades para un buen estimador
		\begin{itemize}
			\item El sesgo de $\hat{y}$ es $E (\hat{y}) - \mu_{Y}$
			\item Sea $\tilde{y}$ o $\tilde{\mu}_{Y}$ otro estimador de $\mu_{Y}$ y supongamos que ambos estimadores son insesgados. Entonces, se dice que $\hat{y}$ es más eficiente que $\tilde{y}$ si var$(\hat{y}) <$ var$(\tilde{y})$
		\end{itemize}
	Tenga en cuenta que un estimador insesgado es consistente pero lo contrario no se cumple.
\end{frame}
%------------------------------------------------
\begin{frame}{Propiedades MELI}
	Algunos ejemplos. Vamos a considere los siguientes estimadores $\hat{\mu}_{Y^{j}}$ de $\mu_{Y}$. Compare los siguientes estimadores en términos de insesgadez y eficiencia.
		\begin{itemize}
			\item $\hat{\mu}_{Y^{0}}= \overline{Y} \equiv \frac{\sum \limits_{i=1}^{n}y_{i}}{n} $
			\item $\hat{\mu}_{Y^{1}}= \sum \limits_{i=1}^{n}y_{i}$
			\item $\hat{\mu}_{Y^{2}}= \sum \limits_{i=0}^{n}\omega^{i}y_{i}$ donde $\omega$ es una ponderación entre 0 y 1.
			\item $\hat{\mu}_{Y^{3}}= \sum \limits_{i=1}^{n}\lambda_{i}y_{i}$ donde $\sum_{i=0}^{n}\lambda_{i}=1$
		\end{itemize}
	Todas las variables aleatorias son i.i.d.(\textit{idéntica e independientemente distribuida})
\end{frame}

%-----------------------------------------------------------------
\subsection{Momentos muestrales}
%-----------------------------------------------------------------
\begin{frame}{Varianza muestral}
	la varianza es calculada como sigue
		$$ s_{Y}^{2} = \frac{\sum \limits_{i=1}^{n}(y_{i}-\overline{Y})^2}{n-1}$$ 
	la varianza de la muestra será útil para probar hipótesis. La desviación estándar de la muestra $(s_{Y})$ es la raíz de la varianza de la muestra
\end{frame}

%-----------------------------------------------------------------
\subsection{Prueba de Hipótesis}
%-----------------------------------------------------------------
\begin{frame}{Ho versus Ha}
	La hipótesis nula $(H_0)$ se especifica para la prueba. La hipótesis alternativa $(H_1 o H_a)$ se cumple si la nula no lo hace. En general, la hipotesis bilateral se configura como
		\begin{align*}
			&H_0 : E(x) = b\\
			&H_1 : E(x) \neq b
		\end{align*}
	La hipótesis unilateral se establece como
		\begin{align*}
			&H_0 : E(x) = b\\
			&H_1 : E(x) < b
		\end{align*}
	donde b es una escalar. En general, para probar la hipótesis nula, reemplazamos la media de la población con una estimación (por ejemplo, $\overline{x}$). Debemos concluir entre rechazar o no rechazar la hipótesis nula.
\end{frame}
%------------------------------------------------
\begin{frame}{Nivel de significancia}
	\begin{itemize}
		\item Es el tamaño de la prueba. Se denota como $\alpha$. Es el área de rechazo propuesta por el investigador.
		\item Eso nos da un umbral para rechazar o no rechazar la hipótesis nula.
		\item No confunda este término con la probabilidad de significancia o el p-value.
		\item Los valores estándar son: 0,01; 0.05 y 0.10
			\begin{itemize}
				\item si eliges valores más altos, más probabilidad de rechazar el nulo. Pero si el nulo es verdadero, tenemos problemas (Error tipo I, denotamos este error como $\alpha$)
				\item si eliges valores más pequeños, mayor probabilidad de no rechazar el nulo. Pero si el nulo es falso, tenemos problemas (Tipo de error II, denotamos este error como $1-\alpha$)
			\end{itemize}
	\end{itemize}
\end{frame}
%------------------------------------------------
\begin{frame}{El p-value}
	\begin{itemize}
		\item Stock y Watson (2011) afirman: ``Es la probabilidad de obtener una estadística al menos tan adversa a la hipótesis nula como la que realmente calculó en su muestra, asumiendo que la hipótesis nula es correcta''.
		\item Como profesor, digo ``Es la probabilidad más pequeña de rechazar la hipótesis nula''.
		\item Como alummno, digo  ``Es la probabilidad de obtener valores superiores a la estimación''.
		\item Debe contrastar el valor p con el nivel de significancia (?). Por lo tanto, tenemos que si 
			$$p-value < \alpha$$
		Entonces debes rechazar la hipótesis nula; de manera equivalente, puede decir ``Hay evidencia en la muestra para rechazar el Ho''
	\end{itemize}
\end{frame}
%------------------------------------------------
\begin{frame}{El p-value}
	Debes considerar la distribución y el tipo de $H_o$ que vas a probar:\\
		\medskip
		{\small{\centering
		\begin{tabular}{ccc} 
			\hline
				Distribución & Lados & P-valor\\
			\hline
					\addlinespace[1ex]
				N		 & Uno & Pr$\left(\frac{X-b}{\sigma_{\hat{x}}} < \frac{\hat{x}-b}{\sigma_{\hat{x}}} \right)$ or Pr$\left(\frac{X-b}{\sigma_{\hat{x}}} > \frac{\hat{x}-b}{\sigma_{\hat{x}}} \right)$\\
					\addlinespace[1ex]
				{}       & Dos & 2$\cdot \left[ \textup{Pr}\left(\frac{X-b}{\sigma_{\hat{x}}} > \left|  \frac{\hat{x}-b}{\sigma_{\hat{x}}}\right| \right)\right] $\\
					\addlinespace[1ex]
			\hline
					\addlinespace[1ex]
				$\chi^2$ & Uno & $\textup{Pr}\left(\chi^{2} > \hat{\chi}^{2} \right) $\\
					\addlinespace[1ex]
			\hline
					\addlinespace[1ex]
				F		 & Uno & $\textup{Pr}\left( F > \hat{F} \right) $\\
					\addlinespace[1ex]
			\hline
					\addlinespace[1ex]
				t		 & Uno & Pr$\left(\frac{X-b}{\sigma_{\hat{x}}} < \frac{\hat{x}-b}{\sigma_{\hat{x}}} \right)$ or Pr$\left(\frac{X-b}{\sigma_{\hat{x}}} > \frac{\hat{x}-b}{\sigma_{\hat{x}}} \right)$\\
					\addlinespace[1ex]
				{}       & Dos & 2$\cdot \left[ \textup{Pr}\left(\frac{X-b}{\sigma_{\hat{x}}} > \left|  \frac{\hat{x}-b}{\sigma_{\hat{x}}}\right| \right)\right] $\\
					\addlinespace[1ex]
			\hline
		\end{tabular}}\\}
	\medskip
	Donde $X$ se distribuye como $N(b,\sigma_{x}^{2})$
\end{frame}
%------------------------------------------------
\begin{frame}{El p-value}
	\begin{figure}
		\centering
		\includegraphics[scale=.30]{figuras/pvalue.png}
	\end{figure}
\end{frame}
%------------------------------------------------
\begin{frame}{Ejemplo}
	\begin{itemize}
		\item Un estudiante de una clase de econometría realiza una prueba estadística en sus calificaciones. Este alumno configuró el siguiente $Ho: \overline{Y} = 70$. La distribución de sus calificaciones -basada en el desempeño pasado- se considera normal con media 50 y varianza 6.
		\item Sean $y_1, y_2 \ldots y_n$ i.i.d se extrae de una distribución con media $\mu_{Y}$. Una prueba de $H_0: \mu_{Y} = 5; H1: \mu_{Y} \neq 5$ utilizando el estadístico t habitual produce un valor p de 0,03. ¿Cuál es la conclusión de la prueba?.
	\end{itemize}
\end{frame}
%------------------------------------------------
\begin{frame}{El p-value}
	\begin{itemize}
		\item ¿Qué es el P-Value?
		\item Definición formal:
				{\centering \textit{``es la probabilidad de obtener un efecto por lo menos tan extremo como el de los datos de la muestra, asumiendo que la hipótesis nula es verdadera.''}}
		\item ¿Se entiende?
		\item Déjenme intentar con una aproximación distinta usando un caso de la vida real.
	\end{itemize}
\end{frame}
%------------------------------------------------
\begin{frame}{El p-value}
	La escena del crimen
	\begin{figure}
		\centering
		\includegraphics[scale=.30]{figuras/pvalue1.png}
	\end{figure}
\end{frame}
%------------------------------------------------
\begin{frame}{El p-value}
	Lo correcto en cualquier juicio es asumir inocencia (Paso 1: hipótesis nula).
	\begin{figure}
		\centering
		\includegraphics[scale=.30]{figuras/pvalue2.png}
	\end{figure}
\end{frame}
%------------------------------------------------
\begin{frame}{El p-value}
	Esto obliga a definir un conjunto de posibilidades en las que el cachorro es inocente aunque algunas de ellas sean poco probables (Paso 2: Modelo matemático)
	\begin{figure}
		\centering
		\includegraphics[scale=.30]{figuras/pvalue3.png}
	\end{figure}
\end{frame}
%------------------------------------------------
\begin{frame}{El p-value}
	Usar la evidencia para concluir algo (Paso 3: Contrastación)
	\begin{figure}
		\centering
		\includegraphics[scale=.30]{figuras/pvalue4.png}
	\end{figure}
\end{frame}
%------------------------------------------------
\begin{frame}{El p-value}
	Analogía con un juicio
		\begin{itemize}
			\item ¿Cuál es la hipótesis en un juicio?
					$$Inocencia\enskip (hipótesis\enskip nula) $$
			\item ¿Cuál es el rol de un ?scal?
					$$Rechazar\enskip la\enskip nula.$$
			\item Pero el ?scal siempre tiene un margen de error. 
			\item La probabilidad de error al tratar de rechazar la nula es el P-Value.
		\end{itemize}
\end{frame}
%------------------------------------------------
\begin{frame}{El p-value}
	El fiscal siempre rechaza la nula, la diferencia está en el p-value
		\begin{itemize}
			\item Si Fiscal señala que el error es de 0.00001\%.
			\item Si Fiscal señala que comete un error de 99.99\%.
		\end{itemize}
\end{frame}
%------------------------------------------------
\begin{frame}{El p-value}
	\begin{itemize}
		\item Y si el fiscal dice que el error es de 20\%...
		\item ¿Cuál es el máximo error que están dispuestos a tolerar para estar de acuerdo con el fiscal? ¿1\%?; ¿5\%?; ¿10\%?
		\item Dicho umbral o tolerancia se le conoce como \textbf{NIVEL DE SIGNIFICANCIA}, es subjetivo y depende del analista y del problema que se esté analizando.
		\item ``No es igual un p-value de 10\% en un experimento médico que en un paper universitario''.
	\end{itemize}
\end{frame}
%------------------------------------------------
\begin{frame}{EL p-value}
	Para concluir algo respecto a una hipótesis sólo se requieren de dos elementos:
		\begin{enumerate}
			\item Conocer la hipótesis
			\item Identificar el p-value asociado a la hipótesis
		\end{enumerate}
	Que este valor sea pequeño o grande depende del analista, aunque es usual compararlo con los niveles de 1\%, 5\% y 10\%.
\end{frame}
%------------------------------------------------
\begin{frame}{Conclusiones}
	Debes tener en cuenta los siguientes pasos para tener una puntuación completa:
		\begin{enumerate}
			\item Escriba la hipótesis nula y alternante.
			\item Elija el nivel de significancia o $\alpha$.
			\item Elija la prueba correctamente (y la distribución relacionada)
			\item Determine el área de rechazo (de acuerdo con la configuración realizada en el paso 1)
			\item Calcule el estadístico.
			\item ¡Y da la conclusión !. (¡PASO muy importante!)
		\end{enumerate}
\end{frame}

%-----------------------------------------------------------------
\subsection{Intervalo de confianza}
\begin{frame}{Un intertvalo de confianza}
	\begin{itemize}
		\item Se construye un intervalo de confianza alrededor de la estimación y ese conjunto indica todos los valores posibles donde se encuentra el valor real de la media de la población o parámetro (o estimación insesgada); por supuesto con alguna probabilidad de ocurrencia.
		\item Las probabilidades estándar son 99, 95 y 90\%.
		\item Un intervalo de confianza no significa precisión, significa confianza donde se encuentra el verdadero parámetro.
		\item En la página 80, Stock y Watson afirman que \textit{``Un intervalo de confianza de dos lados del 95\% para $\mu_{Y}$ es un intervalo construido para que contenga el verdadero valor de $\mu_{Y}$ al 95\% de todas las muestras aleatorias posibles''}.
	\end{itemize}
\end{frame}
%------------------------------------------------
\begin{frame}{Una notación alternativa para intervalos de confianza}
	Aquí una alternativa y asumiendo normalidad
		\begin{align*}
			& 90\% \textup{ I.C. para} \mu_{Y} =[\hat{\mu}_{Y} \pm 1.64 \cdot \hat{\sigma}_{\mu_{Y}}] \equiv [\overline{\mu_{Y}}; \underline{\mu_{Y}}]_{90\%}\\
			& 95\% \textup{ I.C. para} \mu_{Y} =[\hat{\mu}_{Y} \pm 1.96 \cdot \hat{\sigma}_{\mu_{Y}}] \equiv [\overline{\mu_{Y}}; \underline{\mu_{Y}}]_{95\%}\\
			& 99\% \textup{ I.C. para} \mu_{Y} =[\hat{\mu}_{Y} \pm 2.58 \cdot \hat{\sigma}_{\mu_{Y}}] \equiv [\overline{\mu_{Y}}; \underline{\mu_{Y}}]_{99\%}
		\end{align*}
	Pregunta: ¿qué son esos valores 1,64, 1,96 y 2,58 ?. Respuesta: Estimado alumno, son valores críticos.
\end{frame}

%-----------------------------------------------------------------
\subsection{Comparando promedios de poblaciones similares y diferentes}
%-----------------------------------------------------------------
\begin{frame}{Comparando promedios de poblaciones diferentes}
	La $H_0$ y $H_1$ en este caso son:
		\begin{align*}
			& H_0 : \mu_{POP1} - \mu_{POP2} = d \\
			& H_1 : \mu_{POP1} - \mu_{POP2} \neq d
		\end{align*}
	El estadístico $t$ (asumiendo muestras pequeñas) es
		$$t=\frac{\hat{\mu}_{POP1} - \hat{\mu}_{POP2} - d}{\sqrt{\frac{\hat{S}_{POP1}^{2}}{n_{POP1}}+\frac{\hat{s}_{POP1}^{2}}{n_{POP2}}}}$$
\end{frame}
%------------------------------------------------
\begin{frame}{Comparando promedios de poblaciones similares}
	La $H_0$ y $H_1$ en este caso son:
		\begin{align*}
			& H_0 : \mu_{POP1} - \mu_{POP2} = d \\
			& H_1 : \mu_{POP1} - \mu_{POP2} \neq d
		\end{align*}
	En este caso las varianzas son las mismas(Misma población)
	$$t=\frac{\hat{\mu}_{POP1} - \hat{\mu}_{POP2} - d}{\hat{S}\sqrt{\frac{1}{n_{POP1}}+\frac{1}{n_{POP2}}}}$$
	Y el error estandar se calcula como:
	$$\hat{s}^{2}=\frac{1}{n_{POP1} + n_{POP1} -2}\left[ \sum \limits_{i=1}^{n_{POP1}} (y_{POP1, i}-\hat{\mu}_{POP1})^2+
	\sum \limits_{i=1}^{n_{POP2}}(y_{POP2, i}-\hat{\mu}_{POP2})^2 \right]$$
\end{frame}
%3) La estructura de los datos económicos ----------
	%===============================================================================
\section{La estructura de los datos económicos}
%===============================================================================

%-------------------------------------------------------------------------------
\subsection{Datos en serie de tiempo}
%-------------------------------------------------------------------------------
\begin{frame}{Datos en serie de tiempo}
	Es información recolectada en un instante en el tiempo a un conjunto de agentes que pueden ser personas, empresas, países. Ejemplos de este tipo
	de datos son las encuestas de opinión política, la ENAHO, variables macroeconómicas por países en un momento en el tiempo, etc. ``Muchas unidades de observación en un mismo momento en el tiempo''.
\end{frame}
%------------------------------------------------
\begin{frame}{Datos en serie de tiempo - Ejemplo}
	\centering
	{\small Desempleo en la zona euro}\\
	{\tiny (Ratios)} \\
	\smallskip
	\begin{tabular}{ ccc } 
		\hline
		Años & Desempleo & Desempleo juvenil \\
		\hline\hline
		2000 & 8.5 	& n.a  \\
		2001 & 8.1 	& n.a  \\
		2002 & 8.4 	& n.a  \\
		2003 & 8.8 	& n.a  \\
		2004 & 9.0 	& n.a  \\
		2005 & 9.1 	& n.a  \\
		2006 & 8.5 	& n.a  \\
		2007 & 7.6 	& n.a  \\
		2008 & 7.6 	& n.a  \\
		2009 & 9.6 	& 20.2 \\
		2010 & 10.1 & 20.9 \\
		2011 & 10.2 & 20.8 \\
		\hline
	\end{tabular}
\end{frame}

%-------------------------------------------------------------------------------
\subsection{Datos de corte transversal}
%-------------------------------------------------------------------------------
\begin{frame}{Datos de corte transversal}
	Este tipo de información es recolectada a un solo agente en un lapso de tiempo, como por ejemplo el PBI percápita y el gasto del gobierno de 1970 al 2007 de Perú, las ventas de la empresa X durante los últimos 20 años, etc. ``Una unidad de observación en varios momentos en el tiempo''.
\end{frame}
%------------------------------------------------
\begin{frame}{Datos de corte transversal - Ejemplo}
	\centering
	{\small Índice de Corrupción 2006} \\
	\smallskip
	\begin{tabular}{ ccc } 
		\hline
		Individuo & País & IPC puntaje \\
		\hline\hline
		1 & Finlandia & 9.6 \\
		2 & Islandia  & 9.6 \\
		3 & Dinamarca & 9.5 \\
		4 & Chile	  & 7.3 \\
		5 & USA		  & 7.3 \\
		6 & España	  & 6.8 \\
		7 & Perú	  & 3.3 \\
		8 & Ecuador	  & 2.3 \\
		9 & India	  & 3.3 \\
		\vdots & \vdots & \vdots \\
		\hline
	\end{tabular}
\end{frame}

%-------------------------------------------------------------------------------
\subsection{Datos panel}
%-------------------------------------------------------------------------------
\begin{frame}{Datos panel}
	\begin{description}
		\item[Pool data] Es así como se suele conocer a la unión de datos de corte transversal. En la práctica es el proceso de poner `una base sobre
		otra' y entre sus principales ventajas están las de incrementar el tamaño de la muestra y poder estudiar los efectos de determinado
		acontecimiento u política ocurrido en el momento intermedio del suceso: Encuesta en 1990 + Suceso + Encuesta en 1995
		\item[Datos de panel o longitudinales] En este caso las encuestas son aplicadas a los mismos individuos. Así se puede seguir el comportamiento en cuanto a ingresos
		y gastos de varias familias por dos o mas años o también la información macroeconómica de un conjunto de países por un determinado numero de años.
	\end{description}
\end{frame}
%------------------------------------------------
\begin{frame}{Datos panel - Ejemplo}
	\centering
	{\small Desempeño Económico} \\
	{\tiny (Ratios de crecimiento)}\\
	\smallskip
	\begin{tabular}{ cccc } 
		\hline
		Individuo & Año &País & PBI percápita \\
		\hline\hline
		1 & 1990 & USA & 2.32  \\
		1 & 1991 & USA & 0.12  \\
		1 & 1992 & USA & -2.04 \\
		2 & 1990 & ARG & -8.75 \\
		2 & 1991 & ARG & -3.66 \\
		2 & 1992 & ARG & 11.19 \\
		\vdots & \vdots & \vdots & \vdots \\
		2 & 1999 & ARG & -4.67 \\
		3 & 1991 & SPA & 3.56  \\
		\vdots & \vdots & \vdots & \\
		\hline
	\end{tabular}
\end{frame}
%4) La causalidad y la noción de ceteris 
%	paribus en el análisis econométrico ------------
	%===============================================================================
\section{La causalidad y la noción de ceteris paribus en el análisis econométrico}
%===============================================================================

%-------------------------------------------------------------------------------
\subsection{Coeficiente de correlación}
%-------------------------------------------------------------------------------
\begin{frame}{Coeficiente de correlación}
	\begin{itemize}
		\item Mide la asociación lineal entre dos variables.
		\item Poblacionalmente la covarianza se mide como:
		$$Cov(y,x)=E[(y-E(y))(x-E(x))]$$
		\item Muestralmente se estima como:
		$$\sigma_{y,x}=\frac{\sum_{i=1}^T(x_i-\bar{x})(y_i-\bar{y})}{T-1}$$
		\item Correlación muestral
		$$\rho_{y,x}=\frac{\sigma_{y,x}}{\sigma_{y}\sigma_{x}}$$
	\end{itemize}
\end{frame}
%------------------------------------------------
\begin{frame}{Coeficiente de correlación}
	\begin{figure}
		\centering
		\includegraphics[scale=.36]{figuras/cau_y_corr_1.png}
	\end{figure}
\end{frame}
%------------------------------------------------
\begin{frame}{Coeficiente de correlación}
	\begin{figure}
		\centering
		\includegraphics[scale=.36]{figuras/cau_y_corr_2.png}
	\end{figure}
\end{frame}

%------------------------------------------------------------------------------------
% End
%----
	\begin{frame}
		\maketitle
	\end{frame}
%------------------------------------------------------------------------------------
\end{document}		
%====================================================================================