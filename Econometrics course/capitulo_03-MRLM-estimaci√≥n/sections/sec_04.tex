%===============================================================================
\section[Estimación alternativa]{MCO múltiple: Alternativa de estimación}
%===============================================================================
\begin{frame}{Recordando}
	\begin{enumerate}
		\item Si $y=\hat\beta_0+\hat\beta_1 x_1+\hat u=\hat y+\hat u$
		\begin{itemize}
			\item CPO:
			\item $\sum \hat u=0$
			\item $\sum \hat u x_1=0$
		\end{itemize}
		\item Dos ecuaciones, dos incógnitas. Resolviendo (Probar):
		\item $\hat{\beta}_1=\frac{\sum(x_i-\overline{x})(y_i-\overline y)}{\sum(x_i-\overline x)^2}=\frac{\sum(x_i-\overline x)y_i}{\sum(x_i-\overline x)^2}$
	\end{enumerate}
\end{frame}
%---------------------------------------------------
\begin{frame}{Generalizando}
	\begin{enumerate}
		\item Si $y=\hat\beta_0+\hat\beta_1 x_1+\hat\beta_2 x_2+\hat v=\hat y+\hat v$
		\begin{align}
			\sum \hat v & = 0 \\
			\sum \hat v x_1 & = 0 \\
			\sum \hat v x_2 & = 0
		\end{align}
		\item Imagina que:
		\begin{align}
			x_1 = \gamma_0+\gamma_1 x_2+\hat e=\hat{x}_1+\hat e
		\end{align}
		entonces CPO:
		\begin{itemize}
			\item $\sum \hat e=0$
			\item $\sum \hat e x_2=0$
		\end{itemize}
	\end{enumerate}
\end{frame}
%---------------------------------------------------
\begin{frame}{Estimación} De (2):
	\begin{align}
		\sum\hat v x_1 & = 0 \\
		\sum\hat v (\gamma_0+\gamma_1 x_2+\hat e) & = 0 \\
		\sum\hat v \hat e & = 0 \\
		\sum (y-\hat\beta_0-\hat\beta_1 x_1-\hat\beta_2 x_2) \hat e & = 0\\
		\hat{\beta}_1=\frac{\sum y \hat e}{\sum \hat{e}^2}
	\end{align}
\end{frame}
%---------------------------------------------------
\begin{frame}{Un paso más}
	\begin{align}
		Var(\beta_1|x) & = Var(\frac{\sum y \hat e}{\sum \hat{e}^2}|x) \\
		& = \frac{\sum \hat{e}^2 Var (y|x)}{(\sum \hat e^2)^2} \\
		& = \frac{\sigma^2}{\sum \hat{e}^2}
	\end{align}
	De la ecuación (4) se podría calcular el R cuadrado:\\
	$R^2_1=1-SCR_1/SCT_1=1-\sum \hat{e}^2/\sum (x_1-\overline{x})^2$;entonces\\
	$\sum \hat{e}^2=(1-R_1^2)\sum(x_1-\overline{x})^2$,
	entonces:\\
	\begin{align}
		&= \frac{\sigma^2}{SCT_1(1-R_1^2)}
	\end{align}
\end{frame}	
%---------------------------------------------------
\begin{frame}{Ejemplo}
	Por favor, usando STATA ,descargue el conjunto de datos de crecimiento del sitio web de recursos de SW y realice una regresión basada en el siguiente modelo
		\begin{align*}
			Growth_i  = & \enskip  \beta_0 + \beta_1Tradeshare_i + \beta_2YearSchool_i\\
			 		 	& + \beta_3rev\_coups_i + \beta_4Assasinations_i \\
					 	& + \beta_5rgdp60_i
		\end{align*}
	La líneas de código son\\
		\begin{Stata code}{STATA code}
			{\tiny
			\texttt{\textcolor{codeblue}{use} \textcolor{codecrimson}{\textquotedblleft http://wps.aw.com/wps/media/objects/11422/11696965/empirical/empex\_tb/Growth.dta\textquotedblright}, clear}\\
			\texttt{\textcolor{codeblue}{browse}}\\
			\texttt{\textcolor{codeblue}{describe}}\\
			\texttt{\textcolor{codeblue}{reg} growth tradeShare yearsSchool rev\_coups assasinations rgdp60}}
		\end{Stata code}
\end{frame}