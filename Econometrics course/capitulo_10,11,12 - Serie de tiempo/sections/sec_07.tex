%====================================================================================
\section{Elegir el modelo adecuado}
%====================================================================================
\begin{frame}{Elegir el modelo adecuado (1)}
	\begin{itemize}
		\item Los correlogramas deben darnos una primera impresión del proceso que esta detrás de cada serie.
		\item De ser necesario diferenciar la serie.
		\item De todos los candidatos elegir los más ``parsimoniosos''
		\item Por parsimonia se entiende aquellos que expliquen lo mismo usando menos (recuerde los términos calidad y cantidad)
	\end{itemize}
\end{frame}
%---------------------------------------------------
\begin{frame}{Elegir el modelo adecuado (2)}
	\begin{itemize}
		\item Criterios de parsimonía:
		\item \textcolor{red}Error cuadrático medio $= \sum \frac{e^2}{T-k}$
		\item Si hace memoria este es un componente del $R^2$ ajustado.
		\item Criterio de información de Akaike $= e^{\frac{2k}{T}}\sum \frac{e^2}{T}$
		\item Criterio de información de Scharz $=T\left( \frac{K}{T}\right)\sum \frac{e^2}{T}$
		\item En los tres casos lo que queremos es menor error (calidad) con el menor número de explicativas (cantidad)
		\item Por lo que a menor valor mejor es el modelo.
	\end{itemize}
\end{frame}