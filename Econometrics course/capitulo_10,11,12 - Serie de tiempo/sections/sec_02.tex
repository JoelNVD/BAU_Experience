%====================================================================================
\section{¿Por qué estudiar series de tiempo?}
%====================================================================================

\begin{frame}{Porqué estudiar series de tiempo?}
	\begin{itemize}
		\item Para pronosticar
			\begin{itemize}
				\item ¿Cuánto será la inflación el próximo año?
				\item ¿Cuánto será el volumen de ventas de una empresa X?
			\end{itemize}
		\item Para estimar efectos causales dinámicos
			\begin{itemize}
				\item Si el banco central incrementa la tasa de interés, cuál será el impacto inmediato y el efecto sobre la inflación después de dos meses?
			\end{itemize}
	\end{itemize}
\end{frame}
%---------------------------------------------------
\begin{frame}{Ejemplos de modelos de series de tiempo}
	\begin{itemize}
		\item Modelo estático que relaciona variables contemporáneas:
			$$c_t = \beta_0 + \beta_1PBI_t + u_t$$
		\item Modelos que permiten que una o más variables afecten a y con rezagos:
			$$c_t = \alpha_0 + \delta_0 PBI_t + \delta_1 PBI_{t-1} + \delta_2 PBI_{t-2} + u_t$$
		\item \textcolor{red}{Modelos Univariados:}
			$$\textcolor{red}{c_t = \alpha_0 + \delta_0 c_{t-1} + \delta_2 c_{t-2} + \ldots + u_t}$$
		\item En general:
			$$c_t = \alpha_0 + \delta_0 PBI_t + \delta_1 PBI_{t-1} + \delta_2 c_{t-1} + u_t$$
	\end{itemize}
\end{frame}
%---------------------------------------------------
\begin{frame}{Ejemplos de modelos de series de tiempo}
	\begin{itemize}
		\item En lo que queda de clases nos vamos a concentrar en la predicción basado en el pasado de la variable que queremos predecir:
		\item \textcolor{red}{Modelos Univariados:}
			$$\textcolor{red}{c_t = \alpha_0 + \delta_0 c_{t-1} + \delta_2 c_{t-2} + \ldots + u_t}$$
	\end{itemize}
\end{frame}