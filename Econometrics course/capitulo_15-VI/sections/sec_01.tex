%====================================================================================
\section{Introducción}
%====================================================================================

\begin{frame}{Introducción}
	\begin{itemize}
		\item Se dice que el estudio de la endogeneidad es una de las contribuciones fundamentales de la Econometría a la Estadística.
		\item MCO es inconsistente en el modelo: $y_i=x_i'\beta+\mu_i$ si $Cov[x_i,\mu_i]\neq 0$. Este problema se conoce como \emph{endogeneidad} y una de las soluciones es el uso de \emph{variables instrumentales}.
		\item Un instrumento es una variable exógena, es decir: $Cov[z_i,\mu_i]=0$ (condición de exogeneidad) que esta correlacionado con la variable endógena (condición de relevancia) y que por tanto puede ser usado para la estimación del modelo MCO.
	\end{itemize}
\end{frame}