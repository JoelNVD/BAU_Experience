%===============================================================================
\section{Definición del modelo de regresión simple}
%===============================================================================

%-------------------------------------------------------------------------------
\subsection{Terminología}
%-------------------------------------------------------------------------------
\begin{frame}{Terminología}
	\begin{description}
		\item[El modelo] $y=\beta_{0}+\beta_{1}x+u$
		\pause
		\item[$y$] Variable dependiente, variable explicada.
		\pause
		\item[$x$] Variable independiente, regresor, variable explicativa, variable de control o covariado.
		\pause
		\item[$u$] Término de error poblacional.
		\pause
		\item[$\beta_{0}$] Intercepto, constante o coeficiente no asociado con variables.
		\pause
		\item[$\beta_{1}$] Pendiente o coeficiente relacionado a $x$.
	\end{description}
\end{frame}
%---------------------------------------------------
\begin{frame}{Terminología}
	El MRLS es {\large $$ Y_{i} = \beta_{0} + \beta_{1}X_{i}+u_{i}$$}El subíndice $i$ va desde $i = 1$ a $i = n$ (última observación). $Y$ es la variable dependiente; $X$ es la variable dependiente; $\beta_{0}$ es el intercepto, constante o coeficiente no asociado con variables; $\beta_{1}$ es la pendiente o el coeficiente relacionado a $X_{i}$ y $u_{i}$ es el témino error. $Y$ y $X$ son datos.\\
	
	También, podemos denotar $\bar{Y}$ y $\bar{X}$ como el promedio muestral de las variables $Y$ y $X$ respectivamente. En término matriciales, tenemos
	$$ \underrightarrow{Y} = \left [ \begin{array}{c}
		50		\\
		30		\\
		20		\\
		\vdots	\\
		100
	\end{array}
	\right] 
	\quad \textup{y} \quad
	\underrightarrow{X} = \left [ \begin{array}{c}
		2		\\
		4		\\
		1		\\
		\vdots	\\
		6
	\end{array}
	\right]$$
\end{frame}
