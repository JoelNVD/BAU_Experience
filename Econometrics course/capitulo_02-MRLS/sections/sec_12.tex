%===============================================================================
\section{Derivación de estimaciones de mínimos cuadrados ordinarios}
%===============================================================================

%-------------------------------------------------------------------------------	
\subsection{Estimación: Mínimos cuadrados ordinarios (MCO)}
%-------------------------------------------------------------------------------
\begin{frame}{Estimación: Mínimos cuadrados ordinarios}
	Dado que se quiere ajustar un grupo de puntos (muestra) a una línea, lo lógico sería entonces minimizar los errores cometidos por tratar de aproximar los puntos por una recta.
	$$\hat Y = \alpha+ \widehat{\beta} X$$
	Error:
	$$\varepsilon_i=Y_i-\hat Y_i=Y_i-(\alpha+\widehat{\beta} X)$$
	Objetivo es minimizar 
	$$\sum_{i=1}^n \varepsilon _i ^2 = \textup{Min} \sum_{i=1}^{n} (Y_{i}-\hat{Y_{i}})^{2}=\textup{Min} \sum_{i=1}^{n} (Y_{i}-\hat{\beta}_{0}-\hat{\beta}_{1}X_{i})^{2}$$
	¿por qué no minimizar los errores sin ninguna potencia? ¿y con otra potencia? ¿y si se minimizan los valores absolutos?
\end{frame}
%---------------------------------------------------
\begin{frame}{MCO simple}
	\begin{itemize}
		\item Modelo Poblacional
		$$y=\beta_{0}+\beta_{1}x+\mu$$
		\item Modelo muestral
		$$\widehat{y}=\widehat{\beta}_{0}+\widehat{\beta}{1}x+\epsilon$$
		\item Ecuaciones normales: $\sum \epsilon_{i}=0$; $\sum \epsilon_{i}x_{i}=0$
		\item Pendiente
		$$\textcolor{red}{\widehat{\beta}_1 = \frac{\sum (x_{i}-\overline{x})(y_{i}-\overline{y})}{\sum (x_{i}-\overline{x})^2}=\frac{\sum (x_{i}-\overline{x})y_{i}}{\sum (x_{i}-\overline{x})^2}}$$
		\item Intercepto
		$$\textcolor{red}{\widehat{\beta}_{0}=\overline{y}-\widehat{\beta}_{1}\overline{x}}$$
	\end{itemize}
\end{frame}
\begin{frame}{Valores predichos y residuales}
	Una vez tengamos las estimaciones (basados en la data, porsupuesto) podemos calcular los valores predichos ($\widehat{y}$) y los residuos ($\widehat{u}$)
	\begin{gather*}
		\widehat{y}_ = \widehat{\beta}_0 + \widehat{\beta}_1x\\
		\widehat{u} = y - \widehat{y}
	\end{gather*}
	El término de error y los residuales no son lo mismo. El primero está asociado al PGD; este último asociado al estimador (hecho por mortales)
\end{frame}
%---------------------------------------------------
\begin{frame}{¿Cómo obtenemos las estimaciones de MCO}
	Consideremos la siguiente data de corte transversal (notas sobre el cuestionario 1). Descargar la data \href{https://econweb.rutgers.edu/frojas/teaching/undergraduate/grades_data.xlsx}{\textcolor{red}{quí}}
	
	\begin{table}
		\centering
		\begin{tabular}{cccccc}
			i & $y_i$ & $x_i$ & $(x_i - \overline{x})$ & $(x_i - \overline{x})^2$ & $(y_i - \overline{y})$ \\
			\hline
			1 & 4.4 & 3 & {} & {} & {}\\
			2 & 4.1 & 3 & {} & {} & {}\\
			3 & 5.2 & 5 & {} & {} & {}\\
			4 & 2.7 & 3 & {} & {} & {}\\
			$\vdots$ & $\vdots$ & $\vdots$ & $\vdots$ & $\vdots$ & $\vdots$ \\
			80 & 3.1 & 4 & {} & {} & {} \\
			\hline
		\end{tabular}
	\end{table}
\end{frame}
%---------------------------------------------------
\begin{frame}{¿Cómo obtenemos las estimaciones de MCO?}
	Una vez que tengamos las estimaciones para $\beta_0$ y $\beta_2$ podemos completar la siguiente tabla.
	\begin{table}
		\centering
		\begin{tabular}{cccccc}
			i & $y_i$ & $x_i$ & $\beta_0$ & $\beta_1x_i$ & $\widehat{y}_i$ \\
			\hline
			1 & 4.4 & 3 & {} & {} & {}\\
			2 & 4.1 & 3 & {} & {} & {}\\
			3 & 5.2 & 5 & {} & {} & {}\\
			4 & 2.7 & 3 & {} & {} & {}\\
			$\vdots$ & $\vdots$ & $\vdots$ & $\vdots$ & $\vdots$ & $\vdots$ \\
			80 & 3.1 & 4 & {} & {} & {} \\
			\hline
		\end{tabular}
	\end{table}
	Pero, ¿serán insesgado y eficientes?
\end{frame}
%---------------------------------------------------
\begin{frame}{La relación entre $Y$ y $X$}
	Comencemos con el modelo que explica (en términos estadísticos) los determinantes de las calificaciones en nuestra clase de econometría.
	$$Y_i = \beta_0 + \beta_1X_i + u_i$$
	Observe a $\beta_{1}$, acompaña a la variable de interés y nos dice qué tanto cambia $Y$ ante cambios en $X$; es decir  $$\Delta Y = \beta_{1} \cdot \Delta X$$Por lo tanto $$\beta_{1} \rightarrow \textup{ es el objetivo}$$Pero no conocemos al verdadero valor de  $\beta_{1}$ (solo Dios lo sabe); sin embargo, podemos encontrar un estimador $\hat{\beta}_{1}$ (los economistas lo saben) que nos acerque al verdadero valor de $\beta_{1}$\
\end{frame}