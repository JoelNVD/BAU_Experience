%===============================================================================
\section{Teorema de Gauss-Markov}
%===============================================================================

\begin{frame}{Teorema de Gauss-Markov}
	Las condiciones de Gauss-Markov implican 3 supuestos del MCO
		\begin{enumerate}
			\item $E(u_i|X)=0$
			\item $E(u_i^2|X)=\sigma^2$
			\item $E(u_iu_j|X)=0$ para todo $i \neq j$
		\end{enumerate}
\end{frame}

%------------------------------------------------
\begin{frame}{Supuesto \#1}
	Supuesto \#1: La distribución doncidional de $u_i$ dado $X_i$ tienen una media igual a cero.\\
	\medskip
	Eso es
		$$E(u_i|X)=0$$
	Eso implica ($\Longrightarrow$) que la línea de regresión poblacional es la media condicional de $Y_i$ dado $X_i$.
		\begin{align*}
			E(Y_i|X_i) & = E(\beta_0 + \beta_1 X_i + u_i|X_i)\\
					   & = \beta_0 + \beta_1 X_i
		\end{align*}
\end{frame}
%------------------------------------------------
\begin{frame}{Supuesto \#2}
	Supuesto \#2: $(X_i, Y_i), i=1, \ldots , n$. Son independientemente e idénticamente distribuidas.\\
	\medskip
	Asumimos que todas las realizaciones tienen la misma varianza y no están conrrelacionadas una con otra. Formalmente.
		\begin{gather*}
			Var(X_i) = \sigma_{X}^{2}\\
			E(Y_i - \mu_Y)(Y_j - \mu_Y) = 0 \textup{ para todo } i \neq j
		\end{gather*}
	una notación corta para este supuesto es "$i.i.d$". Asumismo ($\Longrightarrow$) lo siguiente
		$$E(u_iu_j|X) = 0 \textup{ para todo } i \neq j$$
\end{frame}
%------------------------------------------------
\begin{frame}{Supuesto \#3}
	Supuesto \#3: Los valores atípicos grandes son poco probables\\
	\medskip
	EL mensaje es que $(X_i, Y_i)$ tiene cuatro momentos finitos distintos de cero (Un coeficiente acotado de curtosis; es decir, $K<\infty$)
\end{frame}
